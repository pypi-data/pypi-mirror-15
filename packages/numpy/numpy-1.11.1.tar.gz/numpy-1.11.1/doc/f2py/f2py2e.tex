\documentclass{article} 
\usepackage{a4wide}
  
\input commands
 
\title{\fpy\\Fortran to Python Interface Generator\\{\large Second Edition}}
\author{Pearu Peterson \texttt{<pearu@ioc.ee>}}
\date{$Revision: 1.16 $\\\today} 
\begin{document}
\special{html: <font size=-1>If equations does not show Greek letters or large
  brackets correctly, then your browser configuration needs some
  adjustment. Read the notes for <A
  href=http://hutchinson.belmont.ma.us/tth/Xfonts.html>Enabling Symbol
  Fonts in Netscape under X </A>. In addition, the browser must be set
  to use document fonts. </font>
}

\maketitle
\begin{abstract}
  \fpy is a Python program that generates Python C/API modules for
  wrapping Fortran~77/90/95 codes to Python. The user can influence the
  process by modifying the signature files that \fpy generates when
  scanning the Fortran codes. This document describes the syntax of
  the signature files and the ways how the user can dictate the tool
  to produce wrapper functions with desired Python signatures. Also
  how to call the wrapper functions from Python is discussed.

  See \texttt{http://cens.ioc.ee/projects/f2py2e/} for updates of this
  document and the tool. 
\end{abstract}

\tableofcontents

\input intro
\input signaturefile
\input notes
\input options
\input bugs

\appendix
\input ex1/foobarmodule
\input apps
\end{document}

%%% Local Variables: 
%%% mode: latex
%%% TeX-master: t
%%% End: 


