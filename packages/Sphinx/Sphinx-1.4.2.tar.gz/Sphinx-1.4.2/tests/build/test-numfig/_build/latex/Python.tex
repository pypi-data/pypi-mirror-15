% Generated by Sphinx.
\def\sphinxdocclass{report}
\documentclass[letterpaper,10pt,english]{sphinxmanual}

\usepackage[utf8]{inputenc}
\ifdefined\DeclareUnicodeCharacter
  \DeclareUnicodeCharacter{00A0}{\nobreakspace}
\else\fi
\usepackage{cmap}
\usepackage[T1]{fontenc}
\usepackage{amsmath,amssymb,amstext}
\usepackage{babel}
\usepackage{times}
\usepackage[Bjarne]{fncychap}
\usepackage{longtable}
\usepackage{sphinx}
\usepackage{multirow}
\usepackage{eqparbox}


\addto\captionsenglish{\renewcommand{\figurename}{Figure:}}
\makeatletter
\def\fnum@figure{\figurename\thefigure.}
\makeatother
\addto\captionsenglish{\renewcommand{\tablename}{Tab\_}}
\makeatletter
\def\fnum@table{\tablename\thetable:}
\makeatother
\SetupFloatingEnvironment{literal-block}{name=Code-}

\addto\extrasenglish{\def\pageautorefname{page}}




\title{Python}
\date{May 25, 2016}
\release{}
\author{}
\newcommand{\sphinxlogo}{}
\renewcommand{\releasename}{Release}
\makeindex

\makeatletter
\def\PYG@reset{\let\PYG@it=\relax \let\PYG@bf=\relax%
    \let\PYG@ul=\relax \let\PYG@tc=\relax%
    \let\PYG@bc=\relax \let\PYG@ff=\relax}
\def\PYG@tok#1{\csname PYG@tok@#1\endcsname}
\def\PYG@toks#1+{\ifx\relax#1\empty\else%
    \PYG@tok{#1}\expandafter\PYG@toks\fi}
\def\PYG@do#1{\PYG@bc{\PYG@tc{\PYG@ul{%
    \PYG@it{\PYG@bf{\PYG@ff{#1}}}}}}}
\def\PYG#1#2{\PYG@reset\PYG@toks#1+\relax+\PYG@do{#2}}

\expandafter\def\csname PYG@tok@gd\endcsname{\def\PYG@tc##1{\textcolor[rgb]{0.63,0.00,0.00}{##1}}}
\expandafter\def\csname PYG@tok@gu\endcsname{\let\PYG@bf=\textbf\def\PYG@tc##1{\textcolor[rgb]{0.50,0.00,0.50}{##1}}}
\expandafter\def\csname PYG@tok@gt\endcsname{\def\PYG@tc##1{\textcolor[rgb]{0.00,0.27,0.87}{##1}}}
\expandafter\def\csname PYG@tok@gs\endcsname{\let\PYG@bf=\textbf}
\expandafter\def\csname PYG@tok@gr\endcsname{\def\PYG@tc##1{\textcolor[rgb]{1.00,0.00,0.00}{##1}}}
\expandafter\def\csname PYG@tok@cm\endcsname{\let\PYG@it=\textit\def\PYG@tc##1{\textcolor[rgb]{0.25,0.50,0.56}{##1}}}
\expandafter\def\csname PYG@tok@vg\endcsname{\def\PYG@tc##1{\textcolor[rgb]{0.73,0.38,0.84}{##1}}}
\expandafter\def\csname PYG@tok@vi\endcsname{\def\PYG@tc##1{\textcolor[rgb]{0.73,0.38,0.84}{##1}}}
\expandafter\def\csname PYG@tok@mh\endcsname{\def\PYG@tc##1{\textcolor[rgb]{0.13,0.50,0.31}{##1}}}
\expandafter\def\csname PYG@tok@cs\endcsname{\def\PYG@tc##1{\textcolor[rgb]{0.25,0.50,0.56}{##1}}\def\PYG@bc##1{\setlength{\fboxsep}{0pt}\colorbox[rgb]{1.00,0.94,0.94}{\strut ##1}}}
\expandafter\def\csname PYG@tok@ge\endcsname{\let\PYG@it=\textit}
\expandafter\def\csname PYG@tok@vc\endcsname{\def\PYG@tc##1{\textcolor[rgb]{0.73,0.38,0.84}{##1}}}
\expandafter\def\csname PYG@tok@il\endcsname{\def\PYG@tc##1{\textcolor[rgb]{0.13,0.50,0.31}{##1}}}
\expandafter\def\csname PYG@tok@go\endcsname{\def\PYG@tc##1{\textcolor[rgb]{0.20,0.20,0.20}{##1}}}
\expandafter\def\csname PYG@tok@cp\endcsname{\def\PYG@tc##1{\textcolor[rgb]{0.00,0.44,0.13}{##1}}}
\expandafter\def\csname PYG@tok@gi\endcsname{\def\PYG@tc##1{\textcolor[rgb]{0.00,0.63,0.00}{##1}}}
\expandafter\def\csname PYG@tok@gh\endcsname{\let\PYG@bf=\textbf\def\PYG@tc##1{\textcolor[rgb]{0.00,0.00,0.50}{##1}}}
\expandafter\def\csname PYG@tok@ni\endcsname{\let\PYG@bf=\textbf\def\PYG@tc##1{\textcolor[rgb]{0.84,0.33,0.22}{##1}}}
\expandafter\def\csname PYG@tok@nl\endcsname{\let\PYG@bf=\textbf\def\PYG@tc##1{\textcolor[rgb]{0.00,0.13,0.44}{##1}}}
\expandafter\def\csname PYG@tok@nn\endcsname{\let\PYG@bf=\textbf\def\PYG@tc##1{\textcolor[rgb]{0.05,0.52,0.71}{##1}}}
\expandafter\def\csname PYG@tok@no\endcsname{\def\PYG@tc##1{\textcolor[rgb]{0.38,0.68,0.84}{##1}}}
\expandafter\def\csname PYG@tok@na\endcsname{\def\PYG@tc##1{\textcolor[rgb]{0.25,0.44,0.63}{##1}}}
\expandafter\def\csname PYG@tok@nb\endcsname{\def\PYG@tc##1{\textcolor[rgb]{0.00,0.44,0.13}{##1}}}
\expandafter\def\csname PYG@tok@nc\endcsname{\let\PYG@bf=\textbf\def\PYG@tc##1{\textcolor[rgb]{0.05,0.52,0.71}{##1}}}
\expandafter\def\csname PYG@tok@nd\endcsname{\let\PYG@bf=\textbf\def\PYG@tc##1{\textcolor[rgb]{0.33,0.33,0.33}{##1}}}
\expandafter\def\csname PYG@tok@ne\endcsname{\def\PYG@tc##1{\textcolor[rgb]{0.00,0.44,0.13}{##1}}}
\expandafter\def\csname PYG@tok@nf\endcsname{\def\PYG@tc##1{\textcolor[rgb]{0.02,0.16,0.49}{##1}}}
\expandafter\def\csname PYG@tok@si\endcsname{\let\PYG@it=\textit\def\PYG@tc##1{\textcolor[rgb]{0.44,0.63,0.82}{##1}}}
\expandafter\def\csname PYG@tok@s2\endcsname{\def\PYG@tc##1{\textcolor[rgb]{0.25,0.44,0.63}{##1}}}
\expandafter\def\csname PYG@tok@nt\endcsname{\let\PYG@bf=\textbf\def\PYG@tc##1{\textcolor[rgb]{0.02,0.16,0.45}{##1}}}
\expandafter\def\csname PYG@tok@nv\endcsname{\def\PYG@tc##1{\textcolor[rgb]{0.73,0.38,0.84}{##1}}}
\expandafter\def\csname PYG@tok@s1\endcsname{\def\PYG@tc##1{\textcolor[rgb]{0.25,0.44,0.63}{##1}}}
\expandafter\def\csname PYG@tok@ch\endcsname{\let\PYG@it=\textit\def\PYG@tc##1{\textcolor[rgb]{0.25,0.50,0.56}{##1}}}
\expandafter\def\csname PYG@tok@m\endcsname{\def\PYG@tc##1{\textcolor[rgb]{0.13,0.50,0.31}{##1}}}
\expandafter\def\csname PYG@tok@gp\endcsname{\let\PYG@bf=\textbf\def\PYG@tc##1{\textcolor[rgb]{0.78,0.36,0.04}{##1}}}
\expandafter\def\csname PYG@tok@sh\endcsname{\def\PYG@tc##1{\textcolor[rgb]{0.25,0.44,0.63}{##1}}}
\expandafter\def\csname PYG@tok@ow\endcsname{\let\PYG@bf=\textbf\def\PYG@tc##1{\textcolor[rgb]{0.00,0.44,0.13}{##1}}}
\expandafter\def\csname PYG@tok@sx\endcsname{\def\PYG@tc##1{\textcolor[rgb]{0.78,0.36,0.04}{##1}}}
\expandafter\def\csname PYG@tok@bp\endcsname{\def\PYG@tc##1{\textcolor[rgb]{0.00,0.44,0.13}{##1}}}
\expandafter\def\csname PYG@tok@c1\endcsname{\let\PYG@it=\textit\def\PYG@tc##1{\textcolor[rgb]{0.25,0.50,0.56}{##1}}}
\expandafter\def\csname PYG@tok@o\endcsname{\def\PYG@tc##1{\textcolor[rgb]{0.40,0.40,0.40}{##1}}}
\expandafter\def\csname PYG@tok@kc\endcsname{\let\PYG@bf=\textbf\def\PYG@tc##1{\textcolor[rgb]{0.00,0.44,0.13}{##1}}}
\expandafter\def\csname PYG@tok@c\endcsname{\let\PYG@it=\textit\def\PYG@tc##1{\textcolor[rgb]{0.25,0.50,0.56}{##1}}}
\expandafter\def\csname PYG@tok@mf\endcsname{\def\PYG@tc##1{\textcolor[rgb]{0.13,0.50,0.31}{##1}}}
\expandafter\def\csname PYG@tok@err\endcsname{\def\PYG@bc##1{\setlength{\fboxsep}{0pt}\fcolorbox[rgb]{1.00,0.00,0.00}{1,1,1}{\strut ##1}}}
\expandafter\def\csname PYG@tok@mb\endcsname{\def\PYG@tc##1{\textcolor[rgb]{0.13,0.50,0.31}{##1}}}
\expandafter\def\csname PYG@tok@ss\endcsname{\def\PYG@tc##1{\textcolor[rgb]{0.32,0.47,0.09}{##1}}}
\expandafter\def\csname PYG@tok@sr\endcsname{\def\PYG@tc##1{\textcolor[rgb]{0.14,0.33,0.53}{##1}}}
\expandafter\def\csname PYG@tok@mo\endcsname{\def\PYG@tc##1{\textcolor[rgb]{0.13,0.50,0.31}{##1}}}
\expandafter\def\csname PYG@tok@kd\endcsname{\let\PYG@bf=\textbf\def\PYG@tc##1{\textcolor[rgb]{0.00,0.44,0.13}{##1}}}
\expandafter\def\csname PYG@tok@mi\endcsname{\def\PYG@tc##1{\textcolor[rgb]{0.13,0.50,0.31}{##1}}}
\expandafter\def\csname PYG@tok@kn\endcsname{\let\PYG@bf=\textbf\def\PYG@tc##1{\textcolor[rgb]{0.00,0.44,0.13}{##1}}}
\expandafter\def\csname PYG@tok@cpf\endcsname{\let\PYG@it=\textit\def\PYG@tc##1{\textcolor[rgb]{0.25,0.50,0.56}{##1}}}
\expandafter\def\csname PYG@tok@kr\endcsname{\let\PYG@bf=\textbf\def\PYG@tc##1{\textcolor[rgb]{0.00,0.44,0.13}{##1}}}
\expandafter\def\csname PYG@tok@s\endcsname{\def\PYG@tc##1{\textcolor[rgb]{0.25,0.44,0.63}{##1}}}
\expandafter\def\csname PYG@tok@kp\endcsname{\def\PYG@tc##1{\textcolor[rgb]{0.00,0.44,0.13}{##1}}}
\expandafter\def\csname PYG@tok@w\endcsname{\def\PYG@tc##1{\textcolor[rgb]{0.73,0.73,0.73}{##1}}}
\expandafter\def\csname PYG@tok@kt\endcsname{\def\PYG@tc##1{\textcolor[rgb]{0.56,0.13,0.00}{##1}}}
\expandafter\def\csname PYG@tok@sc\endcsname{\def\PYG@tc##1{\textcolor[rgb]{0.25,0.44,0.63}{##1}}}
\expandafter\def\csname PYG@tok@sb\endcsname{\def\PYG@tc##1{\textcolor[rgb]{0.25,0.44,0.63}{##1}}}
\expandafter\def\csname PYG@tok@k\endcsname{\let\PYG@bf=\textbf\def\PYG@tc##1{\textcolor[rgb]{0.00,0.44,0.13}{##1}}}
\expandafter\def\csname PYG@tok@se\endcsname{\let\PYG@bf=\textbf\def\PYG@tc##1{\textcolor[rgb]{0.25,0.44,0.63}{##1}}}
\expandafter\def\csname PYG@tok@sd\endcsname{\let\PYG@it=\textit\def\PYG@tc##1{\textcolor[rgb]{0.25,0.44,0.63}{##1}}}

\def\PYGZbs{\char`\\}
\def\PYGZus{\char`\_}
\def\PYGZob{\char`\{}
\def\PYGZcb{\char`\}}
\def\PYGZca{\char`\^}
\def\PYGZam{\char`\&}
\def\PYGZlt{\char`\<}
\def\PYGZgt{\char`\>}
\def\PYGZsh{\char`\#}
\def\PYGZpc{\char`\%}
\def\PYGZdl{\char`\$}
\def\PYGZhy{\char`\-}
\def\PYGZsq{\char`\'}
\def\PYGZdq{\char`\"}
\def\PYGZti{\char`\~}
% for compatibility with earlier versions
\def\PYGZat{@}
\def\PYGZlb{[}
\def\PYGZrb{]}
\makeatother

\renewcommand\PYGZsq{\textquotesingle}

\begin{document}

\maketitle
\tableofcontents
\phantomsection\label{index::doc}



\chapter{Foo}
\label{foo:foo}\label{foo::doc}\label{foo:test-tocdepth}\begin{figure}[htbp]
\centering
\capstart

\includegraphics{{rimg}.png}
\caption{should be Fig.1.1}\label{foo:id1}\end{figure}


\begin{threeparttable}
\capstart\caption{should be Table 1.1}\label{foo:id2}
\begin{tabulary}{\linewidth}{|L|L|}
\hline

hello
 & 
world
\\
\hline\end{tabulary}

\end{threeparttable}


\def\SphinxLiteralBlockLabel{\label{foo:should-be-list-1-1}}
\SphinxSetupCaptionForVerbatim{literal-block}{should be List 1.1}
\begin{Verbatim}[commandchars=\\\{\}]
\PYG{k}{print}\PYG{p}{(}\PYG{l+s+s1}{\PYGZsq{}}\PYG{l+s+s1}{hello world}\PYG{l+s+s1}{\PYGZsq{}}\PYG{p}{)}
\end{Verbatim}
\let\SphinxVerbatimTitle\empty
\let\SphinxLiteralBlockLabel\empty


\section{Foo A}
\label{foo:foo-a}\begin{figure}[htbp]
\centering
\capstart

\includegraphics{{rimg}.png}
\caption{should be Fig.1.2}\label{foo:id3}\end{figure}
\begin{figure}[htbp]
\centering
\capstart

\includegraphics{{rimg}.png}
\caption{should be Fig.1.3}\label{foo:id4}\end{figure}


\begin{threeparttable}
\capstart\caption{should be Table 1.2}\label{foo:id5}
\begin{tabulary}{\linewidth}{|L|L|}
\hline

hello
 & 
world
\\
\hline\end{tabulary}

\end{threeparttable}



\begin{threeparttable}
\capstart\caption{should be Table 1.3}\label{foo:id6}
\begin{tabulary}{\linewidth}{|L|L|}
\hline

hello
 & 
world
\\
\hline\end{tabulary}

\end{threeparttable}


\def\SphinxLiteralBlockLabel{\label{foo:should-be-list-1-2}}
\SphinxSetupCaptionForVerbatim{literal-block}{should be List 1.2}
\begin{Verbatim}[commandchars=\\\{\}]
\PYG{k}{print}\PYG{p}{(}\PYG{l+s+s1}{\PYGZsq{}}\PYG{l+s+s1}{hello world}\PYG{l+s+s1}{\PYGZsq{}}\PYG{p}{)}
\end{Verbatim}
\let\SphinxVerbatimTitle\empty
\let\SphinxLiteralBlockLabel\empty

\def\SphinxLiteralBlockLabel{\label{foo:should-be-list-1-3}}
\SphinxSetupCaptionForVerbatim{literal-block}{should be List 1.3}
\begin{Verbatim}[commandchars=\\\{\}]
\PYG{k}{print}\PYG{p}{(}\PYG{l+s+s1}{\PYGZsq{}}\PYG{l+s+s1}{hello world}\PYG{l+s+s1}{\PYGZsq{}}\PYG{p}{)}
\end{Verbatim}
\let\SphinxVerbatimTitle\empty
\let\SphinxLiteralBlockLabel\empty


\subsection{Foo A1}
\label{foo:foo-a1}

\section{Foo B}
\label{foo:foo-b}

\subsection{Foo B1}
\label{foo:foo-b1}\begin{figure}[htbp]
\centering
\capstart

\includegraphics{{rimg}.png}
\caption{should be Fig.1.4}\label{foo:id7}\end{figure}


\begin{threeparttable}
\capstart\caption{should be Table 1.4}\label{foo:id8}
\begin{tabulary}{\linewidth}{|L|L|}
\hline

hello
 & 
world
\\
\hline\end{tabulary}

\end{threeparttable}


\def\SphinxLiteralBlockLabel{\label{foo:should-be-list-1-4}}
\SphinxSetupCaptionForVerbatim{literal-block}{should be List 1.4}
\begin{Verbatim}[commandchars=\\\{\}]
\PYG{k}{print}\PYG{p}{(}\PYG{l+s+s1}{\PYGZsq{}}\PYG{l+s+s1}{hello world}\PYG{l+s+s1}{\PYGZsq{}}\PYG{p}{)}
\end{Verbatim}
\let\SphinxVerbatimTitle\empty
\let\SphinxLiteralBlockLabel\empty


\chapter{Bar}
\label{bar:bar}\label{bar::doc}

\section{Bar A}
\label{bar:bar-a}\begin{figure}[htbp]
\centering
\capstart

\includegraphics{{rimg}.png}
\caption{should be Fig.2.1}\label{bar:id1}\end{figure}


\begin{threeparttable}
\capstart\caption{should be Table 2.1}\label{bar:id2}
\begin{tabulary}{\linewidth}{|L|L|}
\hline

hello
 & 
world
\\
\hline\end{tabulary}

\end{threeparttable}


\def\SphinxLiteralBlockLabel{\label{bar:should-be-list-2-1}}
\SphinxSetupCaptionForVerbatim{literal-block}{should be List 2.1}
\begin{Verbatim}[commandchars=\\\{\}]
\PYG{k}{print}\PYG{p}{(}\PYG{l+s+s1}{\PYGZsq{}}\PYG{l+s+s1}{hello world}\PYG{l+s+s1}{\PYGZsq{}}\PYG{p}{)}
\end{Verbatim}
\let\SphinxVerbatimTitle\empty
\let\SphinxLiteralBlockLabel\empty


\subsection{Baz A}
\label{baz:baz-a}\label{baz::doc}\begin{figure}[htbp]
\centering
\capstart

\includegraphics{{rimg}.png}
\caption{should be Fig.2.2}\label{baz:fig22}\label{baz:id1}\end{figure}


\begin{threeparttable}
\capstart\caption{should be Table 2.2}\label{baz:table22}\label{baz:id2}
\begin{tabulary}{\linewidth}{|L|L|}
\hline

hello
 & 
world
\\
\hline\end{tabulary}

\end{threeparttable}


\def\SphinxLiteralBlockLabel{\label{baz:code22}\label{baz:should-be-list-2-2}}
\SphinxSetupCaptionForVerbatim{literal-block}{should be List 2.2}
\begin{Verbatim}[commandchars=\\\{\}]
\PYG{k}{print}\PYG{p}{(}\PYG{l+s+s1}{\PYGZsq{}}\PYG{l+s+s1}{hello world}\PYG{l+s+s1}{\PYGZsq{}}\PYG{p}{)}
\end{Verbatim}
\let\SphinxVerbatimTitle\empty
\let\SphinxLiteralBlockLabel\empty
\begin{figure}[htbp]
\centering
\capstart

\includegraphics{{rimg}.png}
\caption{should be Fig.2.3}\label{bar:id3}\end{figure}


\begin{threeparttable}
\capstart\caption{should be Table 2.3}\label{bar:id4}
\begin{tabulary}{\linewidth}{|L|L|}
\hline

hello
 & 
world
\\
\hline\end{tabulary}

\end{threeparttable}


\def\SphinxLiteralBlockLabel{\label{bar:should-be-list-2-3}}
\SphinxSetupCaptionForVerbatim{literal-block}{should be List 2.3}
\begin{Verbatim}[commandchars=\\\{\}]
\PYG{k}{print}\PYG{p}{(}\PYG{l+s+s1}{\PYGZsq{}}\PYG{l+s+s1}{hello world}\PYG{l+s+s1}{\PYGZsq{}}\PYG{p}{)}
\end{Verbatim}
\let\SphinxVerbatimTitle\empty
\let\SphinxLiteralBlockLabel\empty


\section{Bar B}
\label{bar:bar-b}

\subsection{Bar B1}
\label{bar:bar-b1}\begin{figure}[htbp]
\centering
\capstart

\includegraphics{{rimg}.png}
\caption{should be Fig.2.4}\label{bar:id5}\end{figure}


\begin{threeparttable}
\capstart\caption{should be Table 2.4}\label{bar:id6}
\begin{tabulary}{\linewidth}{|L|L|}
\hline

hello
 & 
world
\\
\hline\end{tabulary}

\end{threeparttable}


\def\SphinxLiteralBlockLabel{\label{bar:should-be-list-2-4}}
\SphinxSetupCaptionForVerbatim{literal-block}{should be List 2.4}
\begin{Verbatim}[commandchars=\\\{\}]
\PYG{k}{print}\PYG{p}{(}\PYG{l+s+s1}{\PYGZsq{}}\PYG{l+s+s1}{hello world}\PYG{l+s+s1}{\PYGZsq{}}\PYG{p}{)}
\end{Verbatim}
\let\SphinxVerbatimTitle\empty
\let\SphinxLiteralBlockLabel\empty
\begin{figure}[htbp]
\centering
\capstart

\includegraphics{{rimg}.png}
\caption{should be Fig.1}\label{index:fig1}\label{index:id1}\end{figure}
\begin{figure}[htbp]
\centering
\capstart

\includegraphics{{rimg}.png}
\caption{should be Fig.2}\label{index:id2}\end{figure}


\begin{threeparttable}
\capstart\caption{should be Table 1}\label{index:table-1}\label{index:id3}
\begin{tabulary}{\linewidth}{|L|L|}
\hline

hello
 & 
world
\\
\hline\end{tabulary}

\end{threeparttable}



\begin{threeparttable}
\capstart\caption{should be Table 2}\label{index:id4}
\begin{tabulary}{\linewidth}{|L|L|}
\hline

hello
 & 
world
\\
\hline\end{tabulary}

\end{threeparttable}


\def\SphinxLiteralBlockLabel{\label{index:code-1}\label{index:should-be-list-1}}
\SphinxSetupCaptionForVerbatim{literal-block}{should be List 1}
\begin{Verbatim}[commandchars=\\\{\}]
\PYG{k}{print}\PYG{p}{(}\PYG{l+s+s1}{\PYGZsq{}}\PYG{l+s+s1}{hello world}\PYG{l+s+s1}{\PYGZsq{}}\PYG{p}{)}
\end{Verbatim}
\let\SphinxVerbatimTitle\empty
\let\SphinxLiteralBlockLabel\empty

\def\SphinxLiteralBlockLabel{\label{index:should-be-list-2}}
\SphinxSetupCaptionForVerbatim{literal-block}{should be List 2}
\begin{Verbatim}[commandchars=\\\{\}]
\PYG{k}{print}\PYG{p}{(}\PYG{l+s+s1}{\PYGZsq{}}\PYG{l+s+s1}{hello world}\PYG{l+s+s1}{\PYGZsq{}}\PYG{p}{)}
\end{Verbatim}
\let\SphinxVerbatimTitle\empty
\let\SphinxLiteralBlockLabel\empty
\begin{itemize}
\item {} 
Fig.1 is \hyperref[index:fig1]{Figure:\ref{index:fig1}.}

\item {} 
Fig.2.2 is \hyperref[baz:fig22]{Figure\ref{baz:fig22}}

\item {} 
Table.1 is \hyperref[index:table-1]{Tab\_\ref{index:table-1}:}

\item {} 
Table.2.2 is \hyperref[baz:table22]{Table:\ref{baz:table22}}

\item {} 
List.1 is \hyperref[index:code-1]{Code-\ref{index:code-1} \textbar{} }

\item {} 
List.2.2 is \hyperref[baz:code22]{Code-\ref{baz:code22}}

\item {} 
Invalid numfig\_format 01: \code{invalid}

\item {} 
Invalid numfig\_format 02: \code{Fig \%s \%s}

\end{itemize}



\renewcommand{\indexname}{Index}
\printindex
\end{document}
