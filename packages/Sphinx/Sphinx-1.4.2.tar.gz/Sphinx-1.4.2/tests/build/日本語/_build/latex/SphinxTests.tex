% Generated by Sphinx.
\def\sphinxdocclass{report}
\documentclass[letterpaper,10pt,english]{sphinxmanual}

\usepackage[utf8]{inputenc}
\ifdefined\DeclareUnicodeCharacter
  \DeclareUnicodeCharacter{00A0}{\nobreakspace}
\else\fi
\usepackage{cmap}
\usepackage[T1]{fontenc}
\usepackage{amsmath,amssymb,amstext}
\usepackage{babel}
\usepackage{times}
\usepackage[Bjarne]{fncychap}
\usepackage{longtable}
\usepackage{sphinx}
\usepackage{multirow}
\usepackage{eqparbox}

\addto\captionsenglish{\renewcommand{\contentsname}{Table of Contents}}

\addto\captionsenglish{\renewcommand{\figurename}{Fig. }}
\addto\captionsenglish{\renewcommand{\tablename}{Table }}
\SetupFloatingEnvironment{literal-block}{name=Listing }

\addto\extrasenglish{\def\pageautorefname{page}}

\setcounter{tocdepth}{1}


\title{Sphinx Tests Documentation}
\date{May 25, 2016}
\release{0.6alpha1}
\author{Georg Brandl \and someone else}
\newcommand{\sphinxlogo}{}
\renewcommand{\releasename}{Release}
\makeindex

\makeatletter
\def\PYG@reset{\let\PYG@it=\relax \let\PYG@bf=\relax%
    \let\PYG@ul=\relax \let\PYG@tc=\relax%
    \let\PYG@bc=\relax \let\PYG@ff=\relax}
\def\PYG@tok#1{\csname PYG@tok@#1\endcsname}
\def\PYG@toks#1+{\ifx\relax#1\empty\else%
    \PYG@tok{#1}\expandafter\PYG@toks\fi}
\def\PYG@do#1{\PYG@bc{\PYG@tc{\PYG@ul{%
    \PYG@it{\PYG@bf{\PYG@ff{#1}}}}}}}
\def\PYG#1#2{\PYG@reset\PYG@toks#1+\relax+\PYG@do{#2}}

\expandafter\def\csname PYG@tok@gd\endcsname{\def\PYG@tc##1{\textcolor[rgb]{0.63,0.00,0.00}{##1}}}
\expandafter\def\csname PYG@tok@gu\endcsname{\let\PYG@bf=\textbf\def\PYG@tc##1{\textcolor[rgb]{0.50,0.00,0.50}{##1}}}
\expandafter\def\csname PYG@tok@gt\endcsname{\def\PYG@tc##1{\textcolor[rgb]{0.00,0.27,0.87}{##1}}}
\expandafter\def\csname PYG@tok@gs\endcsname{\let\PYG@bf=\textbf}
\expandafter\def\csname PYG@tok@gr\endcsname{\def\PYG@tc##1{\textcolor[rgb]{1.00,0.00,0.00}{##1}}}
\expandafter\def\csname PYG@tok@cm\endcsname{\let\PYG@it=\textit\def\PYG@tc##1{\textcolor[rgb]{0.25,0.50,0.56}{##1}}}
\expandafter\def\csname PYG@tok@vg\endcsname{\def\PYG@tc##1{\textcolor[rgb]{0.73,0.38,0.84}{##1}}}
\expandafter\def\csname PYG@tok@vi\endcsname{\def\PYG@tc##1{\textcolor[rgb]{0.73,0.38,0.84}{##1}}}
\expandafter\def\csname PYG@tok@mh\endcsname{\def\PYG@tc##1{\textcolor[rgb]{0.13,0.50,0.31}{##1}}}
\expandafter\def\csname PYG@tok@cs\endcsname{\def\PYG@tc##1{\textcolor[rgb]{0.25,0.50,0.56}{##1}}\def\PYG@bc##1{\setlength{\fboxsep}{0pt}\colorbox[rgb]{1.00,0.94,0.94}{\strut ##1}}}
\expandafter\def\csname PYG@tok@ge\endcsname{\let\PYG@it=\textit}
\expandafter\def\csname PYG@tok@vc\endcsname{\def\PYG@tc##1{\textcolor[rgb]{0.73,0.38,0.84}{##1}}}
\expandafter\def\csname PYG@tok@il\endcsname{\def\PYG@tc##1{\textcolor[rgb]{0.13,0.50,0.31}{##1}}}
\expandafter\def\csname PYG@tok@go\endcsname{\def\PYG@tc##1{\textcolor[rgb]{0.20,0.20,0.20}{##1}}}
\expandafter\def\csname PYG@tok@cp\endcsname{\def\PYG@tc##1{\textcolor[rgb]{0.00,0.44,0.13}{##1}}}
\expandafter\def\csname PYG@tok@gi\endcsname{\def\PYG@tc##1{\textcolor[rgb]{0.00,0.63,0.00}{##1}}}
\expandafter\def\csname PYG@tok@gh\endcsname{\let\PYG@bf=\textbf\def\PYG@tc##1{\textcolor[rgb]{0.00,0.00,0.50}{##1}}}
\expandafter\def\csname PYG@tok@ni\endcsname{\let\PYG@bf=\textbf\def\PYG@tc##1{\textcolor[rgb]{0.84,0.33,0.22}{##1}}}
\expandafter\def\csname PYG@tok@nl\endcsname{\let\PYG@bf=\textbf\def\PYG@tc##1{\textcolor[rgb]{0.00,0.13,0.44}{##1}}}
\expandafter\def\csname PYG@tok@nn\endcsname{\let\PYG@bf=\textbf\def\PYG@tc##1{\textcolor[rgb]{0.05,0.52,0.71}{##1}}}
\expandafter\def\csname PYG@tok@no\endcsname{\def\PYG@tc##1{\textcolor[rgb]{0.38,0.68,0.84}{##1}}}
\expandafter\def\csname PYG@tok@na\endcsname{\def\PYG@tc##1{\textcolor[rgb]{0.25,0.44,0.63}{##1}}}
\expandafter\def\csname PYG@tok@nb\endcsname{\def\PYG@tc##1{\textcolor[rgb]{0.00,0.44,0.13}{##1}}}
\expandafter\def\csname PYG@tok@nc\endcsname{\let\PYG@bf=\textbf\def\PYG@tc##1{\textcolor[rgb]{0.05,0.52,0.71}{##1}}}
\expandafter\def\csname PYG@tok@nd\endcsname{\let\PYG@bf=\textbf\def\PYG@tc##1{\textcolor[rgb]{0.33,0.33,0.33}{##1}}}
\expandafter\def\csname PYG@tok@ne\endcsname{\def\PYG@tc##1{\textcolor[rgb]{0.00,0.44,0.13}{##1}}}
\expandafter\def\csname PYG@tok@nf\endcsname{\def\PYG@tc##1{\textcolor[rgb]{0.02,0.16,0.49}{##1}}}
\expandafter\def\csname PYG@tok@si\endcsname{\let\PYG@it=\textit\def\PYG@tc##1{\textcolor[rgb]{0.44,0.63,0.82}{##1}}}
\expandafter\def\csname PYG@tok@s2\endcsname{\def\PYG@tc##1{\textcolor[rgb]{0.25,0.44,0.63}{##1}}}
\expandafter\def\csname PYG@tok@nt\endcsname{\let\PYG@bf=\textbf\def\PYG@tc##1{\textcolor[rgb]{0.02,0.16,0.45}{##1}}}
\expandafter\def\csname PYG@tok@nv\endcsname{\def\PYG@tc##1{\textcolor[rgb]{0.73,0.38,0.84}{##1}}}
\expandafter\def\csname PYG@tok@s1\endcsname{\def\PYG@tc##1{\textcolor[rgb]{0.25,0.44,0.63}{##1}}}
\expandafter\def\csname PYG@tok@ch\endcsname{\let\PYG@it=\textit\def\PYG@tc##1{\textcolor[rgb]{0.25,0.50,0.56}{##1}}}
\expandafter\def\csname PYG@tok@m\endcsname{\def\PYG@tc##1{\textcolor[rgb]{0.13,0.50,0.31}{##1}}}
\expandafter\def\csname PYG@tok@gp\endcsname{\let\PYG@bf=\textbf\def\PYG@tc##1{\textcolor[rgb]{0.78,0.36,0.04}{##1}}}
\expandafter\def\csname PYG@tok@sh\endcsname{\def\PYG@tc##1{\textcolor[rgb]{0.25,0.44,0.63}{##1}}}
\expandafter\def\csname PYG@tok@ow\endcsname{\let\PYG@bf=\textbf\def\PYG@tc##1{\textcolor[rgb]{0.00,0.44,0.13}{##1}}}
\expandafter\def\csname PYG@tok@sx\endcsname{\def\PYG@tc##1{\textcolor[rgb]{0.78,0.36,0.04}{##1}}}
\expandafter\def\csname PYG@tok@bp\endcsname{\def\PYG@tc##1{\textcolor[rgb]{0.00,0.44,0.13}{##1}}}
\expandafter\def\csname PYG@tok@c1\endcsname{\let\PYG@it=\textit\def\PYG@tc##1{\textcolor[rgb]{0.25,0.50,0.56}{##1}}}
\expandafter\def\csname PYG@tok@o\endcsname{\def\PYG@tc##1{\textcolor[rgb]{0.40,0.40,0.40}{##1}}}
\expandafter\def\csname PYG@tok@kc\endcsname{\let\PYG@bf=\textbf\def\PYG@tc##1{\textcolor[rgb]{0.00,0.44,0.13}{##1}}}
\expandafter\def\csname PYG@tok@c\endcsname{\let\PYG@it=\textit\def\PYG@tc##1{\textcolor[rgb]{0.25,0.50,0.56}{##1}}}
\expandafter\def\csname PYG@tok@mf\endcsname{\def\PYG@tc##1{\textcolor[rgb]{0.13,0.50,0.31}{##1}}}
\expandafter\def\csname PYG@tok@err\endcsname{\def\PYG@bc##1{\setlength{\fboxsep}{0pt}\fcolorbox[rgb]{1.00,0.00,0.00}{1,1,1}{\strut ##1}}}
\expandafter\def\csname PYG@tok@mb\endcsname{\def\PYG@tc##1{\textcolor[rgb]{0.13,0.50,0.31}{##1}}}
\expandafter\def\csname PYG@tok@ss\endcsname{\def\PYG@tc##1{\textcolor[rgb]{0.32,0.47,0.09}{##1}}}
\expandafter\def\csname PYG@tok@sr\endcsname{\def\PYG@tc##1{\textcolor[rgb]{0.14,0.33,0.53}{##1}}}
\expandafter\def\csname PYG@tok@mo\endcsname{\def\PYG@tc##1{\textcolor[rgb]{0.13,0.50,0.31}{##1}}}
\expandafter\def\csname PYG@tok@kd\endcsname{\let\PYG@bf=\textbf\def\PYG@tc##1{\textcolor[rgb]{0.00,0.44,0.13}{##1}}}
\expandafter\def\csname PYG@tok@mi\endcsname{\def\PYG@tc##1{\textcolor[rgb]{0.13,0.50,0.31}{##1}}}
\expandafter\def\csname PYG@tok@kn\endcsname{\let\PYG@bf=\textbf\def\PYG@tc##1{\textcolor[rgb]{0.00,0.44,0.13}{##1}}}
\expandafter\def\csname PYG@tok@cpf\endcsname{\let\PYG@it=\textit\def\PYG@tc##1{\textcolor[rgb]{0.25,0.50,0.56}{##1}}}
\expandafter\def\csname PYG@tok@kr\endcsname{\let\PYG@bf=\textbf\def\PYG@tc##1{\textcolor[rgb]{0.00,0.44,0.13}{##1}}}
\expandafter\def\csname PYG@tok@s\endcsname{\def\PYG@tc##1{\textcolor[rgb]{0.25,0.44,0.63}{##1}}}
\expandafter\def\csname PYG@tok@kp\endcsname{\def\PYG@tc##1{\textcolor[rgb]{0.00,0.44,0.13}{##1}}}
\expandafter\def\csname PYG@tok@w\endcsname{\def\PYG@tc##1{\textcolor[rgb]{0.73,0.73,0.73}{##1}}}
\expandafter\def\csname PYG@tok@kt\endcsname{\def\PYG@tc##1{\textcolor[rgb]{0.56,0.13,0.00}{##1}}}
\expandafter\def\csname PYG@tok@sc\endcsname{\def\PYG@tc##1{\textcolor[rgb]{0.25,0.44,0.63}{##1}}}
\expandafter\def\csname PYG@tok@sb\endcsname{\def\PYG@tc##1{\textcolor[rgb]{0.25,0.44,0.63}{##1}}}
\expandafter\def\csname PYG@tok@k\endcsname{\let\PYG@bf=\textbf\def\PYG@tc##1{\textcolor[rgb]{0.00,0.44,0.13}{##1}}}
\expandafter\def\csname PYG@tok@se\endcsname{\let\PYG@bf=\textbf\def\PYG@tc##1{\textcolor[rgb]{0.25,0.44,0.63}{##1}}}
\expandafter\def\csname PYG@tok@sd\endcsname{\let\PYG@it=\textit\def\PYG@tc##1{\textcolor[rgb]{0.25,0.44,0.63}{##1}}}

\def\PYGZbs{\char`\\}
\def\PYGZus{\char`\_}
\def\PYGZob{\char`\{}
\def\PYGZcb{\char`\}}
\def\PYGZca{\char`\^}
\def\PYGZam{\char`\&}
\def\PYGZlt{\char`\<}
\def\PYGZgt{\char`\>}
\def\PYGZsh{\char`\#}
\def\PYGZpc{\char`\%}
\def\PYGZdl{\char`\$}
\def\PYGZhy{\char`\-}
\def\PYGZsq{\char`\'}
\def\PYGZdq{\char`\"}
\def\PYGZti{\char`\~}
% for compatibility with earlier versions
\def\PYGZat{@}
\def\PYGZlb{[}
\def\PYGZrb{]}
\makeatother

\renewcommand\PYGZsq{\textquotesingle}

\begin{document}

\maketitle
\tableofcontents
\phantomsection\label{contents::doc}


Contents:


\chapter{Extension API tests}
\label{extapi:welcome-to-sphinx-tests-s-documentation}\label{extapi::doc}\label{extapi:extension-api-tests}
Testing directives:
\textbf{from function: Foo}\textbf{from class: Bar}

\chapter{Sphinx image handling}
\label{images::doc}\label{images:sphinx-image-handling}
\includegraphics{{img}.png}

\includegraphics{{foo}.png}

\includegraphics{{img1}.png}

\includegraphics{{img}.pdf}

\includegraphics{{foo}.*}

\includegraphics{{simg}.png}

\includegraphics{{svgimg}.pdf}

\includegraphics{{img.foo}.png}


\chapter{Image including source in subdir}
\label{subdir/images:image-including-source-in-subdir}\label{subdir/images::doc}
\includegraphics{{img1}.png}

\includegraphics{{rimg}.png}


\chapter{Including in subdir}
\label{subdir/includes:including-in-subdir}\label{subdir/includes::doc}
\begin{Verbatim}[commandchars=\\\{\}]
\PYG{n+nb}{print}\PYG{p}{(}\PYG{l+s+s2}{\PYGZdq{}}\PYG{l+s+s2}{line 1}\PYG{l+s+s2}{\PYGZdq{}}\PYG{p}{)}
\end{Verbatim}

\begin{Verbatim}[commandchars=\\\{\}]
\PYG{n+nb}{print}\PYG{p}{(}\PYG{l+s+s2}{\PYGZdq{}}\PYG{l+s+s2}{line 2}\PYG{l+s+s2}{\PYGZdq{}}\PYG{p}{)}
\end{Verbatim}

Absolute \code{/img.png} download.

\includegraphics{{img}.png}

This is an include file.


\chapter{Testing downloadable files}
\label{includes::doc}\label{includes:testing-downloadable-files}
Download \code{img.png} here.
Download \code{this} there.
Don't download \code{this}.


\chapter{Test file and literal inclusion}
\label{includes:test-file-and-literal-inclusion}
\includegraphics{{img2}.png}

\includegraphics{{img2}.png}

\begin{Verbatim}[commandchars=\\\{\}]
\PYG{c+c1}{\PYGZsh{} Literally included file using Python highlighting}
\PYG{c+c1}{\PYGZsh{} \PYGZhy{}*\PYGZhy{} coding: utf\PYGZhy{}8 \PYGZhy{}*\PYGZhy{}}

\PYG{n}{foo} \PYG{o}{=} \PYG{l+s+s2}{\PYGZdq{}}\PYG{l+s+s2}{Including Unicode characters: üöä}\PYG{l+s+s2}{\PYGZdq{}}

\PYG{k}{class} \PYG{n+nc}{Foo}\PYG{p}{:}
    \PYG{k}{pass}

\PYG{k}{class} \PYG{n+nc}{Bar}\PYG{p}{:}
    \PYG{k}{def} \PYG{n+nf}{baz}\PYG{p}{(}\PYG{p}{)}\PYG{p}{:}
        \PYG{k}{pass}

\PYG{k}{def} \PYG{n+nf}{bar}\PYG{p}{(}\PYG{p}{)}\PYG{p}{:} \PYG{k}{pass}
\end{Verbatim}

Encoding `utf-8-sig' used for reading included file u'/Users/tkomiya/work/sphinx/tests/build/\textbackslash{}u65e5\textbackslash{}u672c\textbackslash{}u8a9e/wrongenc.inc' seems to be wrong, try giving an :encoding: option


\begin{Verbatim}[commandchars=\\\{\}]
This file is encoded in latin\PYGZhy{}1 but at first read as utf\PYGZhy{}8.

Max Strauß aß in München eine Leberkässemmel.
\end{Verbatim}

This file is encoded in latin-1 but at first read as utf-8.

Max Strauß aß in München eine Leberkässemmel.


\chapter{Literalinclude options}
\label{includes:literalinclude-options}
\begin{Verbatim}[commandchars=\\\{\}]
class Foo:
    pass
\end{Verbatim}

\begin{Verbatim}[commandchars=\\\{\}]
    def baz():
        pass
\end{Verbatim}

\begin{Verbatim}[commandchars=\\\{\},numbers=left,firstnumber=6,stepnumber=1]
class Foo:
    pass
class Bar:
\end{Verbatim}

\begin{Verbatim}[commandchars=\\\{\}]
foo = \PYGZdq{}Including Unicode characters: üöä\PYGZdq{}
\end{Verbatim}

\begin{Verbatim}[commandchars=\\\{\}]
START CODE
\PYGZsh{} Literally included file using Python highlighting
\PYGZsh{} \PYGZhy{}*\PYGZhy{} coding: utf\PYGZhy{}8 \PYGZhy{}*\PYGZhy{}

foo = \PYGZdq{}Including Unicode characters: üöä\PYGZdq{}

class Foo:
    pass

class Bar:
    def baz():
        pass

def bar(): pass
END CODE
\end{Verbatim}

\begin{Verbatim}[commandchars=\\\{\}]
foo = \PYGZdq{}Including Unicode characters: üöä\PYGZdq{}

class Foo:
    pass

class Bar:
    def baz():
        pass

def bar(): pass
\end{Verbatim}

\begin{Verbatim}[commandchars=\\\{\}]
\PYGZsh{} Literally included file using Python highlighting
\PYGZsh{} \PYGZhy{}*\PYGZhy{} coding: utf\PYGZhy{}8 \PYGZhy{}*\PYGZhy{}

foo = \PYGZdq{}Including Unicode characters: üöä\PYGZdq{}
\end{Verbatim}

\begin{Verbatim}[commandchars=\\\{\}]
\PYG{g+gd}{\PYGZhy{}\PYGZhy{}\PYGZhy{} literal\PYGZus{}orig.inc}
\PYG{g+gi}{+++ literal.inc}
\PYG{g+gu}{@@ \PYGZhy{}1,12 +1,12 @@}
 \PYGZsh{} Literally included file using Python highlighting
 \PYGZsh{} \PYGZhy{}*\PYGZhy{} coding: utf\PYGZhy{}8 \PYGZhy{}*\PYGZhy{}
 
\PYG{g+gd}{\PYGZhy{}foo = \PYGZdq{}Including Unicode characters: üöä\PYGZdq{}  \PYGZsh{} This will be changed}
\PYG{g+gi}{+foo = \PYGZdq{}Including Unicode characters: üöä\PYGZdq{}}
 
\PYG{g+gd}{\PYGZhy{}class FooOrig:}
\PYG{g+gi}{+class Foo:}
     pass
 
\PYG{g+gd}{\PYGZhy{}class BarOrig:}
\PYG{g+gi}{+class Bar:}
     def baz():
         pass
 
\end{Verbatim}

\begin{Verbatim}[commandchars=\\\{\}]
Tabs include file test
\PYGZhy{}\PYGZhy{}\PYGZhy{}\PYGZhy{}\PYGZhy{}\PYGZhy{}\PYGZhy{}\PYGZhy{}\PYGZhy{}\PYGZhy{}\PYGZhy{}\PYGZhy{}\PYGZhy{}\PYGZhy{}\PYGZhy{}\PYGZhy{}\PYGZhy{}\PYGZhy{}\PYGZhy{}\PYGZhy{}\PYGZhy{}\PYGZhy{}

The next line has a tab:
\PYGZhy{}\textbar{} \textbar{}\PYGZhy{}
\end{Verbatim}

\begin{Verbatim}[commandchars=\\\{\}]
\PYG{n}{Tabs} \PYG{n}{include} \PYG{n+nb}{file} \PYG{n}{test}
\PYG{o}{\PYGZhy{}}\PYG{o}{\PYGZhy{}}\PYG{o}{\PYGZhy{}}\PYG{o}{\PYGZhy{}}\PYG{o}{\PYGZhy{}}\PYG{o}{\PYGZhy{}}\PYG{o}{\PYGZhy{}}\PYG{o}{\PYGZhy{}}\PYG{o}{\PYGZhy{}}\PYG{o}{\PYGZhy{}}\PYG{o}{\PYGZhy{}}\PYG{o}{\PYGZhy{}}\PYG{o}{\PYGZhy{}}\PYG{o}{\PYGZhy{}}\PYG{o}{\PYGZhy{}}\PYG{o}{\PYGZhy{}}\PYG{o}{\PYGZhy{}}\PYG{o}{\PYGZhy{}}\PYG{o}{\PYGZhy{}}\PYG{o}{\PYGZhy{}}\PYG{o}{\PYGZhy{}}\PYG{o}{\PYGZhy{}}

\PYG{n}{The} \PYG{n+nb}{next} \PYG{n}{line} \PYG{n}{has} \PYG{n}{a} \PYG{n}{tab}\PYG{p}{:}
\PYG{o}{\PYGZhy{}}\PYG{o}{\textbar{}}      \PYG{o}{\textbar{}}\PYG{o}{\PYGZhy{}}
\end{Verbatim}

\begin{Verbatim}[commandchars=\\\{\},numbers=left,firstnumber=6,stepnumber=1]
class Foo:
    pass
\end{Verbatim}

\begin{Verbatim}[commandchars=\\\{\},numbers=left,firstnumber=6,stepnumber=1]
class Foo:
    pass
\end{Verbatim}

\begin{Verbatim}[commandchars=\\\{\},numbers=left,firstnumber=3,stepnumber=1]
foo = \PYGZdq{}Including Unicode characters: üöä\PYGZdq{}
\end{Verbatim}

Test if dedenting before parsing works.

\begin{Verbatim}[commandchars=\\\{\}]
    \PYG{k}{def} \PYG{n+nf}{baz}\PYG{p}{(}\PYG{p}{)}\PYG{p}{:}
        \PYG{k}{pass}
\end{Verbatim}


\chapter{Docutils include with ``literal''}
\label{includes:docutils-include-with-literal}
While not recommended, it should work (and leave quotes alone).

\begin{Verbatim}[commandchars=\\\{\}]
\PYG{n}{Testing} \PYG{l+s+s2}{\PYGZdq{}}\PYG{l+s+s2}{quotes}\PYG{l+s+s2}{\PYGZdq{}} \PYG{o+ow}{in} \PYG{n}{literal} \PYG{l+s+s1}{\PYGZsq{}}\PYG{l+s+s1}{included}\PYG{l+s+s1}{\PYGZsq{}} \PYG{n}{text}\PYG{o}{.}
\end{Verbatim}


\chapter{Testing various markup}
\label{markup:testing-various-markup}\label{markup::doc}

\section{Meta markup}
\label{markup:meta-markup}
\emph{Section author: Georg Brandl}

\emph{Module author: Georg Brandl}

\begin{SphinxShadowBox}
\textbf{TOC}

\medskip

\begin{itemize}
\item {} 
\phantomsection\label{markup:id9}{\hyperref[markup:testing\string-various\string-markup]{\crossref{Testing various markup}}}
\begin{itemize}
\item {} 
\phantomsection\label{markup:id10}{\hyperref[markup:meta\string-markup]{\crossref{Meta markup}}}

\item {} 
\phantomsection\label{markup:id11}{\hyperref[markup:generic\string-rest]{\crossref{Generic reST}}}
\begin{itemize}
\item {} 
\phantomsection\label{markup:id12}{\hyperref[markup:body\string-directives]{\crossref{Body directives}}}

\item {} 
\phantomsection\label{markup:id13}{\hyperref[markup:admonitions]{\crossref{Admonitions}}}

\end{itemize}

\item {} 
\phantomsection\label{markup:id14}{\hyperref[markup:inline\string-markup]{\crossref{Inline markup}}}

\item {} 
\phantomsection\label{markup:id15}{\hyperref[markup:with]{\crossref{With}}}

\item {} 
\phantomsection\label{markup:id16}{\hyperref[markup:tables]{\crossref{Tables}}}

\item {} 
\phantomsection\label{markup:id17}{\hyperref[markup:figures]{\crossref{Figures}}}

\item {} 
\phantomsection\label{markup:id18}{\hyperref[markup:version\string-markup]{\crossref{Version markup}}}

\item {} 
\phantomsection\label{markup:id19}{\hyperref[markup:code\string-blocks]{\crossref{Code blocks}}}

\item {} 
\phantomsection\label{markup:id20}{\hyperref[markup:misc\string-stuff]{\crossref{Misc stuff}}}

\item {} 
\phantomsection\label{markup:id21}{\hyperref[markup:index\string-markup]{\crossref{Index markup}}}

\item {} 
\phantomsection\label{markup:id22}{\hyperref[markup:o\string-some\string-strange\string-characters]{\crossref{Ö... Some strange characters}}}

\item {} 
\phantomsection\label{markup:id23}{\hyperref[markup:only\string-directive]{\crossref{Only directive}}}

\item {} 
\phantomsection\label{markup:id24}{\hyperref[markup:any\string-role]{\crossref{Any role}}}

\end{itemize}

\end{itemize}
\end{SphinxShadowBox}


\section{Generic reST}
\label{markup:generic-rest}
A global substitution (the definition is in rst\_epilog).
\phantomsection\label{markup:label}
\def\SphinxLiteralBlockLabel{\label{markup:label}}
\begin{Verbatim}[commandchars=\\\{\}]
some code
\end{Verbatim}
\let\SphinxLiteralBlockLabel\empty

Option list:
\begin{optionlist}{3cm}
\item [-h]  
help
\item [-{-}help]  
also help
\end{optionlist}

Line block:

\begin{DUlineblock}{0em}
\item[] line1
\item[]
\begin{DUlineblock}{\DUlineblockindent}
\item[] line2
\item[]
\begin{DUlineblock}{\DUlineblockindent}
\item[] line3
\item[]
\begin{DUlineblock}{\DUlineblockindent}
\item[] line4
\end{DUlineblock}
\end{DUlineblock}
\item[] line5
\end{DUlineblock}
\item[] line6
\item[]
\begin{DUlineblock}{\DUlineblockindent}
\item[] line7
\end{DUlineblock}
\end{DUlineblock}


\subsection{Body directives}
\label{markup:body-directives}
\begin{SphinxShadowBox}
\textbf{Title}

\medskip


Topic body.
\end{SphinxShadowBox}

\begin{SphinxShadowBox}
\textbf{Sidebar}

\medskip

~\\
\textbf{Sidebar subtitle}
\smallskip

Sidebar body.
\end{SphinxShadowBox}
\paragraph{Test rubric}
\begin{quote}

Epigraph title

Epigraph body.

\begin{flushright}
---Author
\end{flushright}
\end{quote}
\begin{quote}

Highlights

Highlights body.
\end{quote}
\begin{quote}

Pull-quote

Pull quote body.
\end{quote}

a

b
\begin{alltt}
with some \emph{markup} inside
\end{alltt}


\subsection{Admonitions}
\label{markup:admonitions}\label{markup:admonition-section}
\begin{notice}{note}{My Admonition}

Admonition text.
\end{notice}

\begin{notice}{note}{Note:}
Note text.
\end{notice}

\begin{notice}{warning}{Warning:}
Warning text.
\end{notice}
\phantomsection\label{markup:some-label}
\begin{notice}{tip}{Tip:}
Tip text.
\end{notice}

Indirect hyperlink targets


\section{Inline markup}
\label{markup:inline-markup}\label{markup:other-label}\label{markup:some-label}
\emph{Generic inline markup}

Adding n to test unescaping.
\begin{itemize}
\item {} 
\textbf{\texttt{command\textbackslash{}n}}

\item {} 
\emph{dfn\textbackslash{}n}

\item {} 
\menuselection{guilabel with \accelerator{a}ccelerator and \textbackslash{}n}

\item {} 
\code{kbd\textbackslash{}n}

\item {} 
\emph{\texttt{mailheader\textbackslash{}n}}

\item {} 
\textbf{\texttt{makevar\textbackslash{}n}}

\item {} 
\emph{\texttt{manpage\textbackslash{}n}}

\item {} 
\emph{\texttt{mimetype\textbackslash{}n}}

\item {} 
\emph{\texttt{newsgroup\textbackslash{}n}}

\item {} 
\textbf{\texttt{program\textbackslash{}n}}

\item {} 
\code{regexp\textbackslash{}n}

\item {} 
\menuselection{File \(\rightarrow\) Close\textbackslash{}n}

\item {} 
\menuselection{\accelerator{F}ile \(\rightarrow\) \accelerator{P}rint}

\item {} 
\code{a/\emph{varpart}/b\textbackslash{}n}

\item {} 
\code{print \emph{i}\textbackslash{}n}

\end{itemize}

\emph{Linking inline markup}
\begin{itemize}
\item {} 
\index{Python Enhancement Proposals!PEP 8}\href{https://www.python.org/dev/peps/pep-0008}{\textbf{PEP 8}}

\item {} 
\index{Python Enhancement Proposals!PEP 8}\href{https://www.python.org/dev/peps/pep-0008}{\textbf{Python Enhancement Proposal \#8}}

\item {} 
\index{RFC!RFC 1}\href{https://tools.ietf.org/html/rfc1.html}{\textbf{RFC 1}}

\item {} 
\index{RFC!RFC 1}\href{https://tools.ietf.org/html/rfc1.html}{\textbf{Request for Comments \#1}}

\item {} 
\index{HOME}\index{environment variable!HOME}{\hyperref[objects:envvar\string-HOME]{\crossref{\code{HOME}}}}

\item {} 
{\hyperref[markup:with]{\crossref{\code{with}}}}

\item {} 
{\hyperref[markup:grammar\string-token\string-try_stmt]{\crossref{\code{try statement}}}}

\item {} 
{\hyperref[markup:admonition\string-section]{\crossref{\DUrole{std,std-ref}{Admonitions}}}}

\item {} 
{\hyperref[markup:some\string-label]{\crossref{\DUrole{std,std-ref}{here}}}}

\item {} 
{\hyperref[markup:some\string-label]{\crossref{\DUrole{std,std-ref}{there}}}}

\item {} 
{\hyperref[markup:my\string-figure]{\crossref{\DUrole{std,std-ref}{My caption of the figure}}}}

\item {} 
{\hyperref[markup:my\string-figure\string-name]{\crossref{\DUrole{std,std-ref}{My caption of the figure}}}}

\item {} 
{\hyperref[markup:my\string-table]{\crossref{\DUrole{std,std-ref}{my table}}}}

\item {} 
{\hyperref[markup:my\string-table\string-name]{\crossref{\DUrole{std,std-ref}{my table}}}}

\item {} 
{\hyperref[markup:my\string-code\string-block]{\crossref{\DUrole{std,std-ref}{my ruby code}}}}

\item {} 
{\hyperref[markup:my\string-code\string-block\string-name]{\crossref{\DUrole{std,std-ref}{my ruby code}}}}

\item {} 
\hyperref[markup:my-figure]{Fig. \ref{markup:my-figure}}

\item {} 
\hyperref[markup:my-figure-name]{Fig. \ref{markup:my-figure-name}}

\item {} 
\hyperref[markup:my-table]{Table \ref{markup:my-table}}

\item {} 
\hyperref[markup:my-table-name]{Table \ref{markup:my-table-name}}

\item {} 
\hyperref[markup:my-code-block]{Listing \ref{markup:my-code-block}}

\item {} 
\hyperref[markup:my-code-block-name]{Listing \ref{markup:my-code-block-name}}

\item {} 
{\hyperref[subdir/includes::doc]{\crossref{\DUrole{doc}{Including in subdir}}}}

\item {} 
\code{:download:} is tested in includes.txt

\item {} 
{\hyperref[objects:cmdoption\string-python\string-c]{\crossref{\code{Python -c option}}}}

\item {} 
This used to crash: \code{\&option}

\end{itemize}

Test \textsc{abbr} (abbreviation) and another \textsc{abbr}.

Testing the \index{index}index role, also available with
\index{title!explicit}\index{explicit!title}explicit title.


\section{With}
\label{markup:with}\label{markup:id1}
(Empty section.)


\section{Tables}
\label{markup:tables}

\begin{threeparttable}
\capstart\caption{my table}\label{markup:my-table}\label{markup:my-table-name}
\begin{tabulary}{\linewidth}{|L|p{5cm}|R|}
\hline

1
 & \begin{itemize}
\item {} 
Block elems

\item {} 
In table

\end{itemize}
 & 
x
\\
\hline
2
 & 
Empty cells:
 & \\
\hline\end{tabulary}

\end{threeparttable}



\begin{threeparttable}
\capstart\caption{empty cell in table header}\label{markup:id6}
\begin{tabulary}{\linewidth}{|L|L|}
\hline


 & \textsf{\relax }\\
\hline
1
 & 
2
\\
\hline
3
 & 
4
\\
\hline\end{tabulary}

\end{threeparttable}


Tables with multirow and multicol:

\begin{tabulary}{\linewidth}{|L|L|L|L|L|}
\hline

1
 &  \multicolumn{2}{l|}{
test!
} &  \multicolumn{2}{l|}{ \multirow{2}{*}{
c
}}\\
\cline{1-3} \multirow{2}{*}{\eqparbox{4582511568}{\vspace{.5\baselineskip}
2~\\
y~\\
x~\\
}} & 
col
 & 
col
 &  \multicolumn{2}{l|}{}\\
\cline{2-5} &  \multicolumn{3}{l|}{
multi-column cell
} & 
x
\\
\hline\end{tabulary}


\begin{tabulary}{\linewidth}{|L|}
\hline
 \multirow{2}{*}{
1
}\\
\\
\cline{1-1}\end{tabulary}



\section{Figures}
\label{markup:figures}\begin{figure}[htbp]
\centering
\capstart

\includegraphics{{img}.png}
\caption{My caption of the figure}{\small 
My description paragraph of the figure.

Description paragraph is wraped with legend node.
}\label{markup:my-figure}\label{markup:my-figure-name}\end{figure}
\begin{wrapfigure}{r}{0pt}
\centering
\includegraphics{{rimg}.png}
\caption{figure with align option}\label{markup:id7}\end{wrapfigure}
\begin{wrapfigure}{r}{0.500\linewidth}
\centering
\includegraphics{{rimg}.png}
\caption{figure with align \& figwidth option}\label{markup:id8}\end{wrapfigure}


\section{Version markup}
\label{markup:version-markup}
\DUrole{versionmodified}{New in version 0.6: }Some funny \textbf{stuff}.

\DUrole{versionmodified}{Changed in version 0.6: }Even more funny stuff.

\DUrole{versionmodified}{Deprecated since version 0.6: }Boring stuff.

\DUrole{versionmodified}{New in version 1.2: }First paragraph of versionadded.

\DUrole{versionmodified}{Changed in version 1.2: }First paragraph of versionchanged.

Second paragraph of versionchanged.


\section{Code blocks}
\label{markup:code-blocks}
\def\SphinxLiteralBlockLabel{\label{markup:my-code-block}\label{markup:my-code-block-name}}
\SphinxSetupCaptionForVerbatim{literal-block}{my ruby code}
\begin{Verbatim}[commandchars=\\\{\},numbers=left,firstnumber=1,stepnumber=1]
\PYG{k}{def} \PYG{n+nf}{ruby?}
    \PYG{k+kp}{false}
\PYG{k}{end}
\end{Verbatim}
\let\SphinxVerbatimTitle\empty
\let\SphinxLiteralBlockLabel\empty

\begin{Verbatim}[commandchars=\\\{\}]
import sys

sys.stdout.write(\PYGZsq{}hello world!\PYGZbs{}n\PYGZsq{})
\end{Verbatim}


\section{Misc stuff}
\label{markup:misc-stuff}
Stuff \footnote[1]{\sphinxAtStartFootnote%
Like footnotes.
}

Reference lookup: \phantomsection\label{markup:id3}{\hyperref[markup:ref1]{\crossref{{[}Ref1{]}}}} (defined in another file).
Reference lookup underscore: \phantomsection\label{markup:id4}{\hyperref[markup:ref\string-1]{\crossref{{[}Ref\_1{]}}}}


\strong{See also:}


something, something else, something more
\begin{description}
\item[{\href{http://www.google.com}{Google}}] \leavevmode
For everything.

\end{description}


\begin{itemize}\setlength{\itemsep}{0pt}\setlength{\parskip}{0pt}
\item {} 
This

\item {} 
is

\item {} 
a horizontal

\item {} 
list

\item {} 
with several

\item {} 
items

\end{itemize}
\paragraph{Side note}

This is a side note.

This tests \code{role names in uppercase}.

\begin{center}LICENSE AGREEMENT
\end{center}

Terry Pratchett, Tolkien, Monty Python.

\begin{description}
\item[{änhlich\index{änhlich|textbf}}] \leavevmode\phantomsection\label{markup:term-anhlich}
Dinge

\item[{boson\index{boson|textbf}}] \leavevmode\phantomsection\label{markup:term-boson}
Particle with integer spin.

\item[{\emph{fermion}\index{fermion|textbf}}] \leavevmode\phantomsection\label{markup:term-fermion}
Particle with half-integer spin.

\item[{tauon\index{tauon|textbf}}] \leavevmode\phantomsection\label{markup:term-tauon}\item[{myon\index{myon|textbf}}] \leavevmode\phantomsection\label{markup:term-myon}\item[{electron\index{electron|textbf}}] \leavevmode\phantomsection\label{markup:term-electron}
Examples for fermions.

\item[{über\index{über|textbf}}] \leavevmode\phantomsection\label{markup:term-uber}
Gewisse

\end{description}


\begin{productionlist}
\phantomsection\label{markup:grammar-token-try_stmt}\production{try\_stmt}{ {\hyperref[markup:grammar\string-token\string-try1_stmt]{\crossref{\code{try1\_stmt}}}} \textbar{} {\hyperref[markup:grammar\string-token\string-try2_stmt]{\crossref{\code{try2\_stmt}}}}}
\phantomsection\label{markup:grammar-token-try1_stmt}\production{try1\_stmt}{ ``try'' '':'' \code{suite}}
\productioncont{ (``except'' {[}\code{expression} {[}'','' \code{target}{]}{]} '':'' \code{suite})+}
\productioncont{ {[}''else'' '':'' \code{suite}{]}}
\productioncont{ {[}''finally'' '':'' \code{suite}{]}}
\phantomsection\label{markup:grammar-token-try2_stmt}\production{try2\_stmt}{ ``try'' '':'' \code{suite}}
\productioncont{ ``finally'' '':'' \code{suite}}
\end{productionlist}



\section{Index markup}
\label{markup:index-markup}
\index{entry}\index{entry!pair}\index{pair!entry}\index{entry!double}\index{double!entry}\index{index!entry triple}\index{entry!triple, index}\index{triple!index entry}\index{keyword!with}\index{with!keyword}\index{from|see{to}}\index{fromalso|see{toalso}}
Invalid index markup...

\index{Main|textbf}\index{Other|textbf}\index{entry!pair|textbf}
\index{Main|textbf}Main


\section{Ö... Some strange characters}
\label{markup:olabel}\label{markup:o-some-strange-characters}
Testing öäü...


\section{Only directive}
\label{markup:only-directive}
In LaTeX.

In both.


\section{Any role}
\label{markup:any-role}
Test referencing to {\hyperref[markup:with]{\crossref{\DUrole{std,std-ref,std,std-ref}{headings}}}} and {\hyperref[objects:func_without_body]{\crossref{\code{objects}}}}.
Also {\hyperref[objects:module\string-mod]{\crossref{\code{modules}}}} and {\hyperref[objects:Time]{\crossref{\code{classes}}}}.

More domains:
\begin{itemize}
\item {} 
{\hyperref[objects:bar.baz]{\crossref{\code{JS}}}}

\item {} 
{\hyperref[objects:c.SphinxType]{\crossref{\code{C}}}}

\item {} 
{\hyperref[objects:userdesc\string-myobj]{\crossref{\code{myobj}}}} (user markup)

\item {} 
{\hyperref[objects:_CPPv2N1n5ArrayE]{\crossref{\code{n::Array}}}}

\item {} 
{\hyperref[objects:cmdoption\string-perl\string-c]{\crossref{\code{perl -c}}}}

\end{itemize}


\chapter{Testing object descriptions}
\label{objects:testing-object-descriptions}\label{objects::doc}\index{func\_without\_module() (built-in function)}

\begin{fulllineitems}
\phantomsection\label{objects:func_without_module}\pysiglinewithargsret{\bfcode{func\_without\_module}}{\emph{a}, \emph{b}, \emph{*c}\optional{, \emph{d}}}{}
Does something.

\end{fulllineitems}

\index{func\_without\_body() (built-in function)}

\begin{fulllineitems}
\phantomsection\label{objects:func_without_body}\pysiglinewithargsret{\bfcode{func\_without\_body}}{}{}
\end{fulllineitems}

\index{func\_with\_unknown\_field() (built-in function)}

\begin{fulllineitems}
\phantomsection\label{objects:func_with_unknown_field}\pysiglinewithargsret{\bfcode{func\_with\_unknown\_field}}{}{}
: :

: empty field name:
\begin{quote}\begin{description}
\item[{Field\_name}] \leavevmode
\item[{Field\_name all lower}] \leavevmode
\item[{FIELD\_NAME}] \leavevmode
\item[{FIELD\_NAME ALL CAPS}] \leavevmode
\item[{Field\_Name}] \leavevmode
\item[{Field\_Name All Word Caps}] \leavevmode
\item[{Field\_name}] \leavevmode
\item[{Field\_name First word cap}] \leavevmode
\item[{FIELd\_name}] \leavevmode
\item[{FIELd\_name PARTial caps}] \leavevmode
\end{description}\end{quote}

\end{fulllineitems}



\begin{fulllineitems}
\pysiglinewithargsret{\bfcode{func\_noindex}}{}{}
\end{fulllineitems}

\index{func\_with\_module() (in module foolib)}

\begin{fulllineitems}
\phantomsection\label{objects:foolib.func_with_module}\pysiglinewithargsret{\code{foolib.}\bfcode{func\_with\_module}}{}{}
\end{fulllineitems}


Referring to \code{func with no index}.
Referring to \code{nothing}.
\phantomsection\label{objects:module-mod}\index{mod (module)}\index{func\_in\_module() (in module mod)}

\begin{fulllineitems}
\phantomsection\label{objects:mod.func_in_module}\pysiglinewithargsret{\code{mod.}\bfcode{func\_in\_module}}{}{}
\end{fulllineitems}

\index{Cls (class in mod)}

\begin{fulllineitems}
\phantomsection\label{objects:mod.Cls}\pysigline{\strong{class }\code{mod.}\bfcode{Cls}}~\index{meth1() (mod.Cls method)}

\begin{fulllineitems}
\phantomsection\label{objects:mod.Cls.meth1}\pysiglinewithargsret{\bfcode{meth1}}{}{}
\end{fulllineitems}

\index{meths() (mod.Cls static method)}

\begin{fulllineitems}
\phantomsection\label{objects:mod.Cls.meths}\pysiglinewithargsret{\strong{static }\bfcode{meths}}{}{}
\end{fulllineitems}

\index{attr (mod.Cls attribute)}

\begin{fulllineitems}
\phantomsection\label{objects:mod.Cls.attr}\pysigline{\bfcode{attr}}
\end{fulllineitems}


\end{fulllineitems}

\index{meth2() (mod.Cls method)}

\begin{fulllineitems}
\phantomsection\label{objects:mod.Cls.meth2}\pysiglinewithargsret{\code{Cls.}\bfcode{meth2}}{}{}
\end{fulllineitems}

\index{Error}

\begin{fulllineitems}
\phantomsection\label{objects:errmod.Error}\pysiglinewithargsret{\strong{exception }\code{errmod.}\bfcode{Error}}{\emph{arg1}, \emph{arg2}}{}
\end{fulllineitems}

\index{var (in module mod)}

\begin{fulllineitems}
\phantomsection\label{objects:mod.var}\pysigline{\code{mod.}\bfcode{var}}
\end{fulllineitems}

\index{func\_without\_module2() (built-in function)}

\begin{fulllineitems}
\phantomsection\label{objects:func_without_module2}\pysiglinewithargsret{\bfcode{func\_without\_module2}}{}{{ $\rightarrow$ annotation}}
\end{fulllineitems}



\begin{fulllineitems}
\pysigline{\bfcode{long(parameter,~~~list)}}\pysigline{\bfcode{another~one}}
\end{fulllineitems}

\index{TimeInt (built-in class)}

\begin{fulllineitems}
\phantomsection\label{objects:TimeInt}\pysigline{\strong{class }\bfcode{TimeInt}}
Has only one parameter (triggers special behavior...)
\begin{quote}\begin{description}
\item[{Parameters}] \leavevmode
\textbf{\texttt{moo}} (Moo) -- Moo

\end{description}\end{quote}

\end{fulllineitems}

\index{Time (built-in class)}

\begin{fulllineitems}
\phantomsection\label{objects:Time}\pysiglinewithargsret{\strong{class }\bfcode{Time}}{\emph{hour}, \emph{minute}, \emph{isdst}}{}~\begin{quote}\begin{description}
\item[{Parameters}] \leavevmode\begin{itemize}
\item {} 
\textbf{\texttt{year}} ({\hyperref[objects:TimeInt]{\crossref{\emph{\texttt{TimeInt}}}}}) -- The year.

\item {} 
\textbf{\texttt{minute}} ({\hyperref[objects:TimeInt]{\crossref{\emph{\texttt{TimeInt}}}}}) -- The minute.

\item {} 
\textbf{\texttt{isdst}} -- whether it's DST

\item {} 
\textbf{\texttt{hour}} (\emph{\texttt{DuplicateType}}) -- Some parameter

\item {} 
\textbf{\texttt{hour}} -- Duplicate param.  Should not lead to crashes.

\item {} 
\textbf{\texttt{extcls}} ({\hyperref[objects:mod.Cls]{\crossref{\emph{\texttt{Cls}}}}}) -- A class from another module.

\end{itemize}

\item[{Returns}] \leavevmode
a new {\hyperref[objects:Time]{\crossref{\code{Time}}}} instance

\item[{Return type}] \leavevmode
{\hyperref[objects:Time]{\crossref{Time}}}

\item[{Raises}] \leavevmode
\textbf{\texttt{ValueError}} -- if the values are out of range

\item[{Variables}] \leavevmode\begin{itemize}
\item {} 
\textbf{\texttt{hour}} (\emph{\texttt{int}}) -- like \emph{hour}

\item {} 
\textbf{\texttt{minute}} (\emph{\texttt{int}}) -- like \emph{minute}

\end{itemize}

\end{description}\end{quote}

\end{fulllineitems}



\chapter{C items}
\label{objects:c-items}\index{Sphinx\_DoSomething (C function)}

\begin{fulllineitems}
\phantomsection\label{objects:c.Sphinx_DoSomething}\pysiglinewithargsret{\bfcode{Sphinx\_DoSomething}}{}{}
\end{fulllineitems}

\index{SphinxStruct.member (C member)}

\begin{fulllineitems}
\phantomsection\label{objects:c.SphinxStruct.member}\pysigline{\bfcode{SphinxStruct.member}}
\end{fulllineitems}

\index{SPHINX\_USE\_PYTHON (C macro)}

\begin{fulllineitems}
\phantomsection\label{objects:c.SPHINX_USE_PYTHON}\pysigline{\bfcode{SPHINX\_USE\_PYTHON}}
\end{fulllineitems}

\index{SphinxType (C type)}

\begin{fulllineitems}
\phantomsection\label{objects:c.SphinxType}\pysigline{\bfcode{SphinxType}}
\end{fulllineitems}

\index{sphinx\_global (C variable)}

\begin{fulllineitems}
\phantomsection\label{objects:c.sphinx_global}\pysigline{\bfcode{sphinx\_global}}
\end{fulllineitems}



\chapter{Javascript items}
\label{objects:javascript-items}\index{foo() (built-in function)}

\begin{fulllineitems}
\phantomsection\label{objects:foo}\pysiglinewithargsret{\bfcode{foo}}{}{}
\end{fulllineitems}

\index{bar (global variable or constant)}

\begin{fulllineitems}
\phantomsection\label{objects:bar}\pysigline{\bfcode{bar}}
\end{fulllineitems}

\index{bar.baz() (bar method)}

\begin{fulllineitems}
\phantomsection\label{objects:bar.baz}\pysiglinewithargsret{\code{bar.}\bfcode{baz}}{\emph{href}, \emph{callback}\optional{, \emph{errback}}}{}~\begin{quote}\begin{description}
\item[{Arguments}] \leavevmode\begin{itemize}
\item {} 
\textbf{\texttt{href}} (\emph{\texttt{string}}) -- The location of the resource.

\item {} 
\textbf{\texttt{callback}} -- Get's called with the data returned by the resource.

\end{itemize}

\item[{Throws}] \leavevmode
\textbf{\texttt{InvalidHref}} -- If the \titleref{href} is invalid.

\item[{Returns}] \leavevmode
\titleref{undefined}

\end{description}\end{quote}

\end{fulllineitems}

\index{bar.spam (bar attribute)}

\begin{fulllineitems}
\phantomsection\label{objects:bar.spam}\pysigline{\code{bar.}\bfcode{spam}}
\end{fulllineitems}



\chapter{References}
\label{objects:references}
Referencing {\hyperref[objects:mod.Cls]{\crossref{\code{mod.Cls}}}} or {\hyperref[objects:mod.Cls]{\crossref{\code{mod.Cls}}}} should be the same.

With target: {\hyperref[objects:c.Sphinx_DoSomething]{\crossref{\code{Sphinx\_DoSomething()}}}} (parentheses are handled),
{\hyperref[objects:c.SphinxStruct.member]{\crossref{\code{SphinxStruct.member}}}}, {\hyperref[objects:c.SPHINX_USE_PYTHON]{\crossref{\code{SPHINX\_USE\_PYTHON}}}},
{\hyperref[objects:c.SphinxType]{\crossref{\code{SphinxType *}}}} (pointer is handled), {\hyperref[objects:c.sphinx_global]{\crossref{\code{sphinx\_global}}}}.

Without target: \code{CFunction()}. \code{malloc()}.

{\hyperref[objects:foo]{\crossref{\code{foo()}}}}
{\hyperref[objects:foo]{\crossref{\code{foo()}}}}

{\hyperref[objects:bar]{\crossref{\code{bar}}}}
{\hyperref[objects:bar.baz]{\crossref{\code{bar.baz()}}}}
{\hyperref[objects:bar.baz]{\crossref{\code{bar.baz()}}}}
{\hyperref[objects:bar.baz]{\crossref{\code{baz()}}}}

{\hyperref[objects:bar.baz]{\crossref{\code{bar.baz}}}}


\chapter{Others}
\label{objects:others}\index{environment variable!HOME}

\begin{fulllineitems}
\phantomsection\label{objects:envvar-HOME}\pysigline{\bfcode{HOME}}
\end{fulllineitems}

\index{python command line option!-c command}\index{-c command!python command line option}

\begin{fulllineitems}
\phantomsection\label{objects:cmdoption-python-c}\pysigline{\bfcode{-c}\code{~command}}
\end{fulllineitems}

\index{perl command line option!-c}\index{-c!perl command line option}

\begin{fulllineitems}
\phantomsection\label{objects:cmdoption-perl-c}\pysigline{\bfcode{-c}\code{}}
\end{fulllineitems}

\index{perl command line option!+p}\index{+p!perl command line option}

\begin{fulllineitems}
\phantomsection\label{objects:cmdoption-perl-arg-+p}\pysigline{\bfcode{+p}\code{}}
\end{fulllineitems}

\index{perl command line option!arg}\index{arg!perl command line option}

\begin{fulllineitems}
\phantomsection\label{objects:cmdoption-perl-arg-arg}\pysigline{\bfcode{arg}\code{}}
\end{fulllineitems}


Link to {\hyperref[objects:cmdoption\string-perl\string-arg\string-+p]{\crossref{\code{perl +p}}}} and {\hyperref[objects:cmdoption\string-perl\string-arg\string-arg]{\crossref{\code{arg}}}}
\index{hg command line option!commit}\index{commit!hg command line option}

\begin{fulllineitems}
\phantomsection\label{objects:cmdoption-hg-arg-commit}\pysigline{\bfcode{commit}\code{}}
\end{fulllineitems}

\index{git-commit command line option!-p}\index{-p!git-commit command line option}

\begin{fulllineitems}
\phantomsection\label{objects:cmdoption-git-commit-p}\pysigline{\bfcode{-p}\code{}}
\end{fulllineitems}


Link to {\hyperref[objects:cmdoption\string-hg\string-arg\string-commit]{\crossref{\code{hg commit}}}} and {\hyperref[objects:cmdoption\string-git\string-commit\string-p]{\crossref{\code{git commit -p}}}}.


\chapter{User markup}
\label{objects:user-markup}\index{myobj (userdesc)}

\begin{fulllineitems}
\phantomsection\label{objects:userdesc-myobj}\pysiglinewithargsret{\bfcode{myobj}}{\emph{parameter}}{}
Description of userdesc.

\end{fulllineitems}


Referencing {\hyperref[objects:userdesc\string-myobj]{\crossref{\code{myobj}}}}.


\chapter{CPP domain}
\label{objects:cpp-domain}\index{n::Array (C++ class)}

\begin{fulllineitems}
\phantomsection\label{objects:_CPPv2N1n5ArrayE}\pysigline{\strong{class }\code{n::}\bfcode{Array}}~\index{n::Array::operator{[}{]} (C++ function)}\index{n::Array::operator{[}{]} (C++ function)}

\begin{fulllineitems}
\phantomsection\label{objects:_CPPv2N1n5ArrayixEj}\pysiglinewithargsret{T \&\code{}\bfcode{operator{[}{]}}}{unsigned \emph{j}}{}\phantomsection\label{objects:_CPPv2NK1n5ArrayixEj}\pysiglinewithargsret{\strong{const} T \&\code{}\bfcode{operator{[}{]}}}{unsigned \emph{j}}{ \strong{const}}
\end{fulllineitems}


\end{fulllineitems}



\chapter{File with UTF-8 BOM}
\label{bom::doc}\label{bom:file-with-utf-8-bom}
This file has a UTF-8 ``BOM''.


\chapter{Test math extensions \protect\(E = m c^2\protect\)}
\label{math:test-math-extensions}\label{math::doc}
This is inline math: \(a^2 + b^2 = c^2\).
\begin{equation*}
\begin{split}a^2 + b^2 = c^2\end{split}
\end{equation*}\begin{equation*}
\begin{split}a + 1 < b\end{split}
\end{equation*}\phantomsection\label{math:equation-foo}\begin{equation}\label{math-foo}
\begin{split}e^{i\pi} = 1\end{split}
\end{equation}\begin{equation*}
\begin{split}e^{ix} = \cos x + i\sin x\end{split}
\end{equation*}\begin{equation*}
\begin{split}n \in \mathbb N\end{split}
\end{equation*}a + 1 < b
Referencing equation \eqref{math-foo}.


\chapter{Autodoc tests}
\label{autodoc:autodoc-tests}\label{autodoc::doc}
Just testing a few autodoc possibilities...
\phantomsection\label{autodoc:module-util}\index{util (module)}

\section{Sphinx test suite utilities}
\label{autodoc:sphinx-test-suite-utilities}\begin{quote}\begin{description}
\item[{copyright}] \leavevmode
Copyright 2007-2016 by the Sphinx team, see AUTHORS.

\item[{license}] \leavevmode
BSD, see LICENSE for details.

\end{description}\end{quote}
\phantomsection\label{autodoc:module-test_autodoc}\index{test\_autodoc (module)}

\section{test\_autodoc}
\label{autodoc:test-autodoc}
Test the autodoc extension.  This tests mainly the Documenters; the auto
directives are tested in a test source file translated by test\_build.
\begin{quote}\begin{description}
\item[{copyright}] \leavevmode
Copyright 2007-2016 by the Sphinx team, see AUTHORS.

\item[{license}] \leavevmode
BSD, see LICENSE for details.

\end{description}\end{quote}
\index{Class (class in test\_autodoc)}

\begin{fulllineitems}
\phantomsection\label{autodoc:test_autodoc.Class}\pysiglinewithargsret{\strong{class }\code{test\_autodoc.}\bfcode{Class}}{\emph{arg}}{}
Class to document.
\index{attr (test\_autodoc.Class attribute)}

\begin{fulllineitems}
\phantomsection\label{autodoc:test_autodoc.Class.attr}\pysigline{\bfcode{attr}\strong{ = `bar'}}
should be documented -- süß

\end{fulllineitems}

\index{descr (test\_autodoc.Class attribute)}

\begin{fulllineitems}
\phantomsection\label{autodoc:test_autodoc.Class.descr}\pysigline{\bfcode{descr}}
Descriptor instance docstring.

\end{fulllineitems}

\index{docattr (test\_autodoc.Class attribute)}

\begin{fulllineitems}
\phantomsection\label{autodoc:test_autodoc.Class.docattr}\pysigline{\bfcode{docattr}\strong{ = `baz'}}
should likewise be documented -- süß

\end{fulllineitems}

\index{excludemeth() (test\_autodoc.Class method)}

\begin{fulllineitems}
\phantomsection\label{autodoc:test_autodoc.Class.excludemeth}\pysiglinewithargsret{\bfcode{excludemeth}}{}{}
Method that should be excluded.

\end{fulllineitems}

\index{inst\_attr\_comment (test\_autodoc.Class attribute)}

\begin{fulllineitems}
\phantomsection\label{autodoc:test_autodoc.Class.inst_attr_comment}\pysigline{\bfcode{inst\_attr\_comment}\strong{ = None}}
a documented instance attribute

\end{fulllineitems}

\index{inst\_attr\_inline (test\_autodoc.Class attribute)}

\begin{fulllineitems}
\phantomsection\label{autodoc:test_autodoc.Class.inst_attr_inline}\pysigline{\bfcode{inst\_attr\_inline}\strong{ = None}}
an inline documented instance attr

\end{fulllineitems}

\index{inst\_attr\_string (test\_autodoc.Class attribute)}

\begin{fulllineitems}
\phantomsection\label{autodoc:test_autodoc.Class.inst_attr_string}\pysigline{\bfcode{inst\_attr\_string}\strong{ = None}}
a documented instance attribute

\end{fulllineitems}

\index{mdocattr (test\_autodoc.Class attribute)}

\begin{fulllineitems}
\phantomsection\label{autodoc:test_autodoc.Class.mdocattr}\pysigline{\bfcode{mdocattr}\strong{ = \textless{}StringIO.StringIO instance\textgreater{}}}
should be documented as well - süß

\end{fulllineitems}

\index{meth() (test\_autodoc.Class method)}

\begin{fulllineitems}
\phantomsection\label{autodoc:test_autodoc.Class.meth}\pysiglinewithargsret{\bfcode{meth}}{}{}
Function.

\end{fulllineitems}

\index{moore() (test\_autodoc.Class class method)}

\begin{fulllineitems}
\phantomsection\label{autodoc:test_autodoc.Class.moore}\pysiglinewithargsret{\strong{classmethod }\bfcode{moore}}{\emph{a}, \emph{e}, \emph{f}}{{ $\rightarrow$ happiness}}
\end{fulllineitems}

\index{prop (test\_autodoc.Class attribute)}

\begin{fulllineitems}
\phantomsection\label{autodoc:test_autodoc.Class.prop}\pysigline{\bfcode{prop}}
Property.

\end{fulllineitems}

\index{skipmeth() (test\_autodoc.Class method)}

\begin{fulllineitems}
\phantomsection\label{autodoc:test_autodoc.Class.skipmeth}\pysiglinewithargsret{\bfcode{skipmeth}}{}{}
Method that should be skipped.

\end{fulllineitems}

\index{udocattr (test\_autodoc.Class attribute)}

\begin{fulllineitems}
\phantomsection\label{autodoc:test_autodoc.Class.udocattr}\pysigline{\bfcode{udocattr}\strong{ = `quux'}}
should be documented as well - süß

\end{fulllineitems}


\end{fulllineitems}

\index{function() (in module test\_autodoc)}

\begin{fulllineitems}
\phantomsection\label{autodoc:test_autodoc.function}\pysiglinewithargsret{\code{test\_autodoc.}\bfcode{function}}{\emph{foo}, \emph{*args}, \emph{**kwds}}{}
Return spam.

\end{fulllineitems}

\index{Class (class in test\_autodoc)}

\begin{fulllineitems}
\pysiglinewithargsret{\strong{class }\code{test\_autodoc.}\bfcode{Class}}{\emph{arg}}{}
Class to document.

Additional content.
\index{attr (test\_autodoc.Class attribute)}

\begin{fulllineitems}
\pysigline{\bfcode{attr}\strong{ = `bar'}}
should be documented -- süß

\end{fulllineitems}

\index{descr (test\_autodoc.Class attribute)}

\begin{fulllineitems}
\pysigline{\bfcode{descr}}
Descriptor instance docstring.

\end{fulllineitems}

\index{docattr (test\_autodoc.Class attribute)}

\begin{fulllineitems}
\pysigline{\bfcode{docattr}\strong{ = `baz'}}
should likewise be documented -- süß

\end{fulllineitems}

\index{excludemeth() (test\_autodoc.Class method)}

\begin{fulllineitems}
\pysiglinewithargsret{\bfcode{excludemeth}}{}{}
Method that should be excluded.

\end{fulllineitems}

\index{inheritedmeth() (test\_autodoc.Class method)}

\begin{fulllineitems}
\phantomsection\label{autodoc:test_autodoc.Class.inheritedmeth}\pysiglinewithargsret{\bfcode{inheritedmeth}}{}{}
Inherited function.

\end{fulllineitems}

\index{inst\_attr\_comment (test\_autodoc.Class attribute)}

\begin{fulllineitems}
\pysigline{\bfcode{inst\_attr\_comment}\strong{ = None}}
a documented instance attribute

\end{fulllineitems}

\index{inst\_attr\_inline (test\_autodoc.Class attribute)}

\begin{fulllineitems}
\pysigline{\bfcode{inst\_attr\_inline}\strong{ = None}}
an inline documented instance attr

\end{fulllineitems}

\index{inst\_attr\_string (test\_autodoc.Class attribute)}

\begin{fulllineitems}
\pysigline{\bfcode{inst\_attr\_string}\strong{ = None}}
a documented instance attribute

\end{fulllineitems}

\index{mdocattr (test\_autodoc.Class attribute)}

\begin{fulllineitems}
\pysigline{\bfcode{mdocattr}\strong{ = \textless{}StringIO.StringIO instance\textgreater{}}}
should be documented as well - süß

\end{fulllineitems}

\index{meth() (test\_autodoc.Class method)}

\begin{fulllineitems}
\pysiglinewithargsret{\bfcode{meth}}{}{}
Function.

\end{fulllineitems}

\index{moore() (test\_autodoc.Class class method)}

\begin{fulllineitems}
\pysiglinewithargsret{\strong{classmethod }\bfcode{moore}}{\emph{a}, \emph{e}, \emph{f}}{{ $\rightarrow$ happiness}}
\end{fulllineitems}

\index{prop (test\_autodoc.Class attribute)}

\begin{fulllineitems}
\pysigline{\bfcode{prop}}
Property.

\end{fulllineitems}

\index{skipmeth() (test\_autodoc.Class method)}

\begin{fulllineitems}
\pysiglinewithargsret{\bfcode{skipmeth}}{}{}
Method that should be skipped.

\end{fulllineitems}

\index{udocattr (test\_autodoc.Class attribute)}

\begin{fulllineitems}
\pysigline{\bfcode{udocattr}\strong{ = `quux'}}
should be documented as well - süß

\end{fulllineitems}


\end{fulllineitems}

\index{Outer (class in test\_autodoc)}

\begin{fulllineitems}
\phantomsection\label{autodoc:test_autodoc.Outer}\pysigline{\strong{class }\code{test\_autodoc.}\bfcode{Outer}}
Foo
\index{Outer.Inner (class in test\_autodoc)}

\begin{fulllineitems}
\phantomsection\label{autodoc:test_autodoc.Outer.Inner}\pysigline{\strong{class }\bfcode{Inner}}
Foo
\index{meth() (test\_autodoc.Outer.Inner method)}

\begin{fulllineitems}
\phantomsection\label{autodoc:test_autodoc.Outer.Inner.meth}\pysiglinewithargsret{\bfcode{meth}}{}{}
Foo

\end{fulllineitems}


\end{fulllineitems}


\end{fulllineitems}

\index{docattr (test\_autodoc.Class attribute)}

\begin{fulllineitems}
\pysigline{\code{Class.}\bfcode{docattr}\strong{ = `baz'}}
should likewise be documented -- süß

\end{fulllineitems}

\index{CustomEx}

\begin{fulllineitems}
\phantomsection\label{autodoc:test_autodoc.CustomEx}\pysigline{\strong{exception }\code{test\_autodoc.}\bfcode{CustomEx}}
My custom exception.
\index{f() (test\_autodoc.CustomEx method)}

\begin{fulllineitems}
\phantomsection\label{autodoc:test_autodoc.CustomEx.f}\pysiglinewithargsret{\bfcode{f}}{}{}
Exception method.

\end{fulllineitems}


\end{fulllineitems}

\index{CustomDict (class in test\_autodoc)}

\begin{fulllineitems}
\phantomsection\label{autodoc:test_autodoc.CustomDict}\pysigline{\strong{class }\code{test\_autodoc.}\bfcode{CustomDict}}
Bases: \code{dict}

Docstring.

\end{fulllineitems}

\index{MarkupError (class in autodoc\_fodder)}

\begin{fulllineitems}
\phantomsection\label{autodoc:autodoc_fodder.MarkupError}\pysigline{\strong{class }\code{autodoc\_fodder.}\bfcode{MarkupError}}~
\begin{notice}{note}{Note:}
This is a docstring with a
\end{notice}

Explicit markup ends without a blank line; unexpected unindent.


small markup error which should have
correct location information.

\end{fulllineitems}

\index{InstAttCls (class in test\_autodoc)}

\begin{fulllineitems}
\phantomsection\label{autodoc:test_autodoc.InstAttCls}\pysigline{\strong{class }\code{test\_autodoc.}\bfcode{InstAttCls}}
Class with documented class and instance attributes.

All members (5 total)
\index{ca1 (test\_autodoc.InstAttCls attribute)}

\begin{fulllineitems}
\phantomsection\label{autodoc:test_autodoc.InstAttCls.ca1}\pysigline{\bfcode{ca1}\strong{ = `a'}}
Doc comment for class attribute InstAttCls.ca1.
It can have multiple lines.

\end{fulllineitems}

\index{ca2 (test\_autodoc.InstAttCls attribute)}

\begin{fulllineitems}
\phantomsection\label{autodoc:test_autodoc.InstAttCls.ca2}\pysigline{\bfcode{ca2}\strong{ = `b'}}
Doc comment for InstAttCls.ca2. One line only.

\end{fulllineitems}

\index{ca3 (test\_autodoc.InstAttCls attribute)}

\begin{fulllineitems}
\phantomsection\label{autodoc:test_autodoc.InstAttCls.ca3}\pysigline{\bfcode{ca3}\strong{ = `c'}}
Docstring for class attribute InstAttCls.ca3.

\end{fulllineitems}

\index{ia1 (test\_autodoc.InstAttCls attribute)}

\begin{fulllineitems}
\phantomsection\label{autodoc:test_autodoc.InstAttCls.ia1}\pysigline{\bfcode{ia1}\strong{ = None}}
Doc comment for instance attribute InstAttCls.ia1

\end{fulllineitems}

\index{ia2 (test\_autodoc.InstAttCls attribute)}

\begin{fulllineitems}
\phantomsection\label{autodoc:test_autodoc.InstAttCls.ia2}\pysigline{\bfcode{ia2}\strong{ = None}}
Docstring for instance attribute InstAttCls.ia2.

\end{fulllineitems}


\end{fulllineitems}

\index{InstAttCls (class in test\_autodoc)}

\begin{fulllineitems}
\pysigline{\strong{class }\code{test\_autodoc.}\bfcode{InstAttCls}}
Class with documented class and instance attributes.

Specific members (2 total)
\index{ca1 (test\_autodoc.InstAttCls attribute)}

\begin{fulllineitems}
\pysigline{\bfcode{ca1}\strong{ = `a'}}
Doc comment for class attribute InstAttCls.ca1.
It can have multiple lines.

\end{fulllineitems}

\index{ia1 (test\_autodoc.InstAttCls attribute)}

\begin{fulllineitems}
\pysigline{\bfcode{ia1}\strong{ = None}}
Doc comment for instance attribute InstAttCls.ia1

\end{fulllineitems}


\end{fulllineitems}

\phantomsection\label{autodoc:module-autodoc_missing_imports}\index{autodoc\_missing\_imports (module)}
\begin{SphinxShadowBox}
\textbf{Dedication}

\medskip


For Docutils users \& co-developers.
\end{SphinxShadowBox}

\begin{SphinxShadowBox}
\textbf{Abstract}

\medskip


This document is a demonstration of the reStructuredText markup
language, containing examples of all basic reStructuredText
constructs and many advanced constructs.
\end{SphinxShadowBox}


\chapter{reStructuredText Demonstration}
\label{metadata:restructuredtext-demonstration}\label{metadata::doc}

\section{Examples of Syntax Constructs}
\label{metadata:examples-of-syntax-constructs}

\chapter{Test for diverse extensions}
\label{extensions::doc}\label{extensions:test-for-diverse-extensions}

\section{extlinks}
\label{extensions:extlinks}
Test diverse links: \href{http://bugs.python.org/issue1000}{issue 1000} and \url{http://python.org/dev/}, also with
\href{http://bugs.python.org/issue1042}{explicit caption}.


\section{todo}
\label{extensions:todo}

\subsection{list of all todos}
\label{extensions:list-of-all-todos}

\chapter{Testing footnote and citation}
\label{footnote:testing-footnote-and-citation}\label{footnote::doc}

\section{numbered footnote}
\label{footnote:numbered-footnote}
\footnote[1]{\sphinxAtStartFootnote%
numbered
}


\section{auto-numbered footnote}
\label{footnote:auto-numbered-footnote}
\footnote[2]{\sphinxAtStartFootnote%
auto numbered
}


\section{named footnote}
\label{footnote:named-footnote}
\footnote[3]{\sphinxAtStartFootnote%
named
}


\section{citation}
\label{footnote:citation}
\phantomsection\label{footnote:id4}{\hyperref[footnote:bar]{\crossref{{[}bar{]}}}}


\section{footnotes in table}
\label{footnote:footnotes-in-table}

\begin{threeparttable}
\capstart\caption{Table caption \protect\footnotemark[4]}\label{footnote:id12}
\begin{tabulary}{\linewidth}{|L|L|}
\hline
\textsf{\relax 
name \protect\footnotemark[5]
} & \textsf{\relax 
desription
}\\
\hline
VIDIOC\_CROPCAP
 & 
Information about VIDIOC\_CROPCAP
\\
\hline\end{tabulary}

\end{threeparttable}

\footnotetext[4]{\sphinxAtStartFootnote%
footnotes in table caption
}\footnotetext[5]{\sphinxAtStartFootnote%
footnotes in table
}

\section{footenotes}
\label{footnote:footenotes}\paragraph{Citations}


\section{missing target}
\label{footnote:missing-target}
{[}missing{]} citation


\chapter{Various kinds of lists}
\label{lists::doc}\label{lists:various-kinds-of-lists}

\section{nested enumerated lists}
\label{lists:nested-enumerated-lists}\begin{enumerate}
\item {} 
one

\item {} 
two
\begin{enumerate}
\item {} 
two.1

\item {} 
two.2

\end{enumerate}

\item {} 
three

\end{enumerate}


\section{enumerated lists with non-default start values}
\label{lists:enumerated-lists-with-non-default-start-values}\begin{enumerate}
\setcounter{enumi}{-1}
\item {} 
zero

\item {} 
one

\end{enumerate}


\bigskip\hrule{}\bigskip

\begin{enumerate}
\item {} 
one

\item {} 
two

\end{enumerate}


\bigskip\hrule{}\bigskip

\begin{enumerate}
\setcounter{enumi}{1}
\item {} 
two

\item {} 
three

\end{enumerate}


\section{enumerated lists using letters}
\label{lists:enumerated-lists-using-letters}\begin{enumerate}
\item {} 
a

\item {} 
b

\item {} 
c

\item {} 
d

\end{enumerate}


\bigskip\hrule{}\bigskip

\begin{enumerate}
\setcounter{enumi}{23}
\item {} 
x

\item {} 
y

\item {} 
z

\item {} 
\{

\end{enumerate}


\section{definition lists}
\label{lists:definition-lists}\begin{description}
\item[{term1}] \leavevmode
description

\item[{term2 (\textbf{stronged partially})}] \leavevmode
description

\end{description}


\chapter{Generated section}
\label{otherext::doc}\label{otherext:id1}

\chapter{Indices and tables}
\label{contents:indices-and-tables}\begin{itemize}
\item {} 
\DUrole{xref,std,std-ref}{genindex}

\item {} 
\DUrole{xref,std,std-ref}{modindex}

\item {} 
\DUrole{xref,std,std-ref}{search}

\end{itemize}


\chapter{References}
\label{contents:references}

\chapter{Test for issue \#1157}
\label{contents:test-for-issue-1157}
This used to crash:


\chapter{Test for issue \#1700}
\label{contents:test-for-issue-1700}
{\hyperref[contents:mastertoc]{\crossref{\DUrole{std,std-ref}{Table of Contents}}}}


\chapter{Test for indirect hyperlink targets}
\label{contents:test-for-indirect-hyperlink-targets}
{\hyperref[markup:some\string-label]{\crossref{\DUrole{std,std-ref}{indirect hyperref}}}}

toctree contains reference to nonexisting document u'\textbackslash{}u65e5\textbackslash{}u672c\textbackslash{}u8a9e/\textbackslash{}u65e5\textbackslash{}u672c\textbackslash{}u8a9e'


\begin{thebibliography}{Ref_1}
\bibitem[bar]{bar}{\phantomsection\label{footnote:bar} 
cite
}
\bibitem[Ref1]{Ref1}{\phantomsection\label{contents:ref1} 
Reference target.
}
\bibitem[Ref\_1]{Ref_1}{\phantomsection\label{contents:ref-1} 
Reference target 2.
}
\end{thebibliography}


\renewcommand{\indexname}{Python Module Index}
\begin{theindex}
\def\bigletter#1{{\Large\sffamily#1}\nopagebreak\vspace{1mm}}
\bigletter{a}
\item {\texttt{autodoc\_missing\_imports}}, \pageref{autodoc:module-autodoc_missing_imports}
\indexspace
\bigletter{m}
\item {\texttt{mod}} \emph{(UNIX)}, \pageref{objects:module-mod}
\indexspace
\bigletter{t}
\item {\texttt{test\_autodoc}}, \pageref{autodoc:module-test_autodoc}
\indexspace
\bigletter{u}
\item {\texttt{util}}, \pageref{autodoc:module-util}
\end{theindex}

\renewcommand{\indexname}{Index}
\printindex
\end{document}
